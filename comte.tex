%!TEX ROOT=filosofia.tex

\section{Comte}
Auguste Comte è il fondatore del positivismo in Francia, nonché fondatore/ideatore della sociologia.
La sua opera principale è \textit{'Corso di filosofia positiva'}. 

\subsection{Legge dei tre stadi}
Comte pensa di aver fatto una scoperta: la \textbf{Legge dei tre stadi} che è una filosofia della
storia e della conoscenza. Sono tre stadi che valgono per l'umanita e il singolo. I seguenti
sono i tre stadi
\begin{description}
  \item[Stadio Teologico] l'uomo è guidato dalla \textbf{fantasia e immaginazione}. Le cause dei
    fenomeni sono soprannaturali. L'epoca storica di riferimento è il \textbf{Medioevo} dominato
    da re e principi, dalla Chiesa e dalla religione. La forma di governo è la \textbf{Monarchia}.
    È un'epoca \textbf{organica}, caratterizzata da ordine e stabilità.
  \item[Stadio Metafisico] è guidato dalla \textbf{ragione} che cerca le cause dei fenomeni, oltre
    i fenomeni (essenza, forma, \ldots). Corrisponde all'\textbf{Età moderna}. È un'epoca 
    \textbf{critica}, caratterizzata da rivoluzioni, cambiamenti, disordini.
  \item[Stadio positivo] ancora non è realizzato, Comte se lo aspetta in un futuro prossimo (visione
    finalistica della storia). La \textbf{scienza} guida il popolo. Perché si arrivi a questo stadio
    (che è \textbf{organico}) è necessario che esista la \textbf{sociologia}. Al potere saranno
    i tecnici che governeranno per il bene comune applicando leggi scientifiche. Il potere
    culturale lo hanno gli scienziati, non è democrazia (le idee fondanti della democrazia sono
    metafisiche, astratte. L'uomo non deve pensare ai diritti, ma ai doveri della società. Gli uomini
    non sono liberi o uguali, la scienza è una sola).
\end{description}

\subsection{Le scienze}
Ogni scienza identifica delle \textbf{leggi} a partire dall'osservazione dei fenomeni. Queste leggi
sono ciò che rende utile la scienza perché consentono di prevedere i fenomeni futuri.\\
Comte voleva arrivare ad una \textbf{classificazione delle scienze}. Uno dei caratteri fondamentali
è la specializzazione delle scienze: Comte non è contrario a questa pratica ma teme si possa perdere
la visione d'insieme. Proprio questo è il compito della filosofia. \textbf{La filosofia deve capire
un metodo scientifico mantenendo la visione generale}. Per Comte si raggiunge la scientificità di una
pratica tanto prima tanto è più generale. E più è generale più è facile. Secondo Comte l'ordine è
Matematica, Fisica, Astronomia, Chimica, Biologia, \ldots. Mancano però due cose: \textbf{Psicologia
e Sociologia}.

\subsubsection{Psicologia}
La psicologia non potrà mai diventare una scienza perché dovrebbe essere basata 
sull'auto-osservazione. Viene quindi meno l'oggettività necessaria per una scienza. Ci sono già delle
scienze che studiano l'uomo: la Biologia e la Sociologia

\subsubsection{Sociologia}
Per capire l'uomo bisogna conoscere la società in quanto l'individuo ne fa parte. Si può dividere in
due
\begin{description}
  \item[Statica] cosa permette la stabilità della società (proprietà privata, famiglia, potere)
  \item[Dinamica] cosa permette il progresso della società (la legge dei tre stadi)
\end{description}

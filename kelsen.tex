%!TEX ROOT=filosofia.tex

\section{Kelsen}
Kelsen è uno dei più grandi filosofi del diritto del '900. Collabora alla scrittura della 
costituzione austriaca, ha l'idea di una corte costituzionale.

\subsection{``Teoria generale del diritto dello stato''}
Kelsen segue il \textbf{positivismo giuridico} ovvero una dottrina scientifica del diritto. Partecipa
infatti al \textbf{circolo di Vienna} dove si discute del \textbf{Neopositivismo}. La scienza è un
saperea autentico ed è \textbf{descrittivo} (descrive i fatti), non dà ``giudizi di valore'' (come
fa la metafisica), irrazionali.\\ [\baselineskip]
L'oggetto di studio è il diritto positivo (quello dello stato, no giusnaturalismo). Non ci sono
valori assoluti, il giurista si occupa di ciò che esiste davvero. Conta solo la \textbf{validità} di
una legge.
\begin{description}
  \item[Legge] Una proposizione prescrittiva (che dà ordini) creata dagli uomini
  \item[Diritto] Dice cosa (non) fare e cosa accade in caso contrario. È eteronomo, c'è un'autorità
    esterna
  \item[Morale] Autonoma, non prevede pene
\end{description}
\textbf{Non ci sono leggi giuste o ingiuste, solo valide o meno}, il compito della giustizia è quello
di validare le leggi, ovvero determinare se è \textbf{coerente} (deducibile) da altre leggi più 
generali. Le norme più particolari sono le sentenze dei giudici (che valgono solo per l'imputato).
È valida solo se c'è una legge che prevede quel provvedimento. Non potendo andare avanti all'infinito
ci deve essere una legge che sia coerente con sè stessa: la \textbf{costituzione}. Tutte le leggi
devono essere coerenti con la costituzione che è fondata sui valori di chi la scrive.

\subsubsection{Lo stato}
Lo stato è \textbf{l'ordinamento giuridico}. L'essere cittadino dipende dalle leggi, quindi lo stato
non sono i cittadini, ma le leggi.\\
Critica la visione marxista dello stato. \textbf{In comune ha che lo stato è} un potere 
\textbf{coercitivo} (per Marx deve mantenere il potere, per Kelsen minacciare punizioni). 
\textbf{In disaccordo} invece è che Marx crede che le classi siano dominate dallo stato. Kelsen 
invece che \textbf{la legge} (e quindi lo stato) \textbf{si possa fare in molti modi} dato che 
possono avere molti scopi. Marx inoltre crede che lo stato si estinguerà (nel comunismo autentico), 
Kelsen invece no. Critica i presupposti: \textbf{la società comunista non sarà necessariamente ricca 
e capace di distribuire i beni} e \textbf{i comportamenti asociali non derivano esclusivamente 
dall'alienazione}. Sarà sempre necessario uno stato.

\subsection{``Essenza e valore della democrazia''}
La democrazia è fondata sulla libertà come valore etico e politico (Rousseau). Ci sono due tipi di
democrazia
\begin{description}
  \item[Ideale] Pura, è l'ideale puro di libertà, irrealizzabile.\\
    \textbf{Democrazia diretta}, assenza di differenza tra governanti e governati, tutti partecipano
    alla formulaizone delle leggi e quindi dello stato. \textbf{Unanimità}, solo così uno è libero,
    tutti sono d'accordo e nessuno è costretto ad obbedire ad un potere esterno (non si possono fare
    discorsi razionali su visioni diverse del mondo).
  \item[Reale] Rinuncia ad alcune parti dell'ideale.\\
    \textbf{Maggioranza} e quindi rinuncia all'unanimità. La maggioranza sarà autonoma, la minoranza
    eteronoma e quindi non accetta le leggi, sarà obbligata. \textbf{Rappresentativa}, non è potere
    al popolo, molti sono esclusi comunque (bambini, donne, \ldots), inoltre non è praticabile far
    partecipare tutti. Il potere è sempre oligarchico.
\end{description}
L'autocrazia è diversa dalla democrazia non per il numero di governanti. Nella democrazia i capi 
sono eletti, in una dittatura sono nominati. Ci sono buoni motivi per scegliere la democrazia però:
\begin{itemize}
  \item Consente il \textbf{ricambio della classe politica} tramite le elezioni (garantisce la
    minoranza)
  \item Controlla chi detiene il potere
\end{itemize}
I partiti e il parlamento sono alla base della democrazia. Il \textbf{parlamento} è una parte dei
cittadini, eletta. Il suo obiettivo è \textbf{trovare un compromesso politico} positivo, che non
minacci i valori. Deve rappresentare il più possibile la società e quindi Kelsen sostiene il
sistema proporzionale e il vincolo di mandato.\\
Kelsen definisce la democrazia \textbf{procedurale}, ovvero puramente formale. È un metodo per 
scegliere la rappresentanza. \textbf{È democratica una legge formulata secondo queste procedure}, 
conta la forma, non la sostanza. Ad esempio se una legge abolisce la proprietà privata, è comunque
democratica anche se ha idee socialiste.\\
L'atteggiamento relativistico porta alla democrazia (è più disposto ad accettare compromessi),
quello assolutistico invece alla autocrazia (dove non c'è tolleranza). \textbf{Non ci sono più valori
condivisi in assoluto}, riprende un po' Mill e il ``politeismo dei valori''.

\subsection{Differenze con Bobbio}
Bobbio è un kelseniano che però ha un'idea diversa sulla democrazia. È sì una procedura ma non basta.
Non solo la forma ma anche il contenuto delle leggi deve essere democratico, deve seguire i principi
fondamentali di libertà.

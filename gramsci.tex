%!TEX ROOT=filosofia.tex

\section{Gramsci}
Nei suoi ``Quaderni del carcere'' definisce la sua idea. \textbf{Era marxista} e voleva opporsi ai
massimalisti e ai riformisti del PSI perché riteneva fossero influenzati dal positivismo (il 
socialismo come evoluzione dal capitalismo). Sia i massimalisti che i riformisti aspettavano. Gramsci
voleva fare, sosteneva il \textbf{volontarismo}. La storia dipende dalle azioni e dalle volontà
degli uomini comuni.\\
Sostiene i consigli di fabbrica in quanto sono \textbf{democrazia autentica} (non borghese), anche
chiamata \textbf{democrazia operaia}. È diretta, non rappresentativa. Sono uno strumento contro il
capitalismo ma sono anche espressione di una democrazia diversa. I rapporti interpersonali cambiano,
non c'è più concorrenza, l'individuo è parte di un tutto, hanno in comune il bene comune, non del 
singolo.\\ [\baselineskip]
Dopo il Fascismo e la fondazione del PCI, Gramsci cambia idea. Il \textbf{partito} è il centro, non
i consigli di fabbrica. I partito è il nuovo principe machiavelliano. Non ci sarà più uno stato, la
democrazia sarà autentica, niente divisioni in classi, emerge la socialità dell'uomo e i 
\textbf{principi morali} nascono spontaneamente.

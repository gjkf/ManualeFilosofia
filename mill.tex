%!TEX ROOT=filosofia.tex

\section{Mill}
John Stuart Mill � un positivista inglese il cui principale scritto � \textit{Sistema di logica 
deduttiva ed induttiva}. 

\subsection{Sistema di logica deduttiva ed induttiva}
Mill � un \textbf{empirista radicale}, si rif� a Locke e a Hume: ogni cosa, anche la pi� astratta,
deriva dall'esperienza.\\
Mill in particolar modo si occupa dell'induzione: in un modo o nell'altro
si deve partire dall'esperienza, anche le premesse di un sillogismo lo fanno. Cosa ci permette di
passare dal particolare all'universale? \textbf{L'uniformit� della natura}. Ovvero che a cause simili
corrispondono effetti simili. Questo non � dato a priori, si ricava anch'esso dall'esperienza. Si
pu� qui entrare in un un circolo vizioso: dall'induzione si trova il principio di causa che trova
l'induzione e cos� via. Fin'ora per� non � mai stato smentito che una causa c'� sempre. Per�
\textbf{non � possibile escludere che ci siano fatti indeterminati}. La scienza quindi � fallibile. 

\subsection{Etica}
Mill � un \textbf{utilitarista} radicale (benesserismo, consequenzialismo, aggregazionismo). Mill
per� si allontana un po' dall'utilitarismo classico in quanto ritiene che ci siano diversi tipi di
piaceri, di diverse qualit�. \textbf{L'altruismo � il piacere pi� alto di tutti}. Questo dimostra
che l'utilitarismo pu� andare senza problemi assieme al Cristianesimo.

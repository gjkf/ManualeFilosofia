%!TEX ROOT=filosofia.tex

\section{Mill}
John Stuart Mill è un positivista inglese il cui principale scritto è \textit{``Sistema di logica 
deduttiva ed induttiva''}. 

\subsection{Sistema di logica deduttiva ed induttiva}
Mill è un \textbf{empirista radicale}, si rifà a Locke e a Hume: ogni cosa, anche la più astratta,
deriva dall'esperienza.\\
Mill in particolar modo si occupa dell'induzione: in un modo o nell'altro
si deve partire dall'esperienza, anche le premesse di un sillogismo lo fanno. Cosa ci permette di
passare dal particolare all'universale? \textbf{L'uniformità della natura}. Ovvero che a cause simili
corrispondono effetti simili. Questo non è dato a priori, si ricava anch'esso dall'esperienza. Si
può qui entrare in un un circolo vizioso: dall'induzione si trova il principio di causa che trova
l'induzione e così via. Fin'ora però non è mai stato smentito che una causa c'è sempre. Però
\textbf{non è possibile escludere che ci siano fatti indeterminati}. La scienza quindi è fallibile. 

\subsection{Etica}
Mill è un \textbf{utilitarista} radicale (benesserismo, consequenzialismo, aggregazionismo). Mill
però si allontana un po' dall'utilitarismo classico in quanto ritiene che ci siano diversi tipi di
piaceri, di diverse qualità. \textbf{L'altruismo è il piacere più alto di tutti}. Questo dimostra
che l'utilitarismo può andare senza problemi assieme al Cristianesimo.\\
L'uomo deve essere lasciato libero di agire ed essere felice fino a che le conseguenze delle sue 
azioni non ricadono su altri individui. Solo in questo caso lo Stato può limitare la libertà. 
\textbf{Sul proprio pensiero e sul proprio corpo, l'individuo è sovrano}. In argomenti di bio-etica
si mantiene la stessa linea di pensiero: se non danneggia altri, si è liberi. Nell'aborto l'embrione
non è razionale, l'unico essere razionale è la madre, quindi l'aborto è consentito.

\subsection{Politica}
Mill è un \textbf{liberale} secondo cui si deve rispettare la legge per evitare di danneggiare una
minoranza. È un male necessario, però le leggi devono lasciare la massima libertà.
\begin{itemize}
  \item Un po' limita la libertà individuale
  \item È necessario per evitare di danneggiare altri
\end{itemize}
Lo Stato dev'essere coercitivo solo per evitare che gli uomini si danneggino fra di loro.\\
Mill crede che \textbf{ci debba essere libertà di religione e di idee} in maniera assoluta. Il 
progresso della storia deriva dalla libertà individuale, le idee dei singoli devono essere espresse,
altrimenti non ci sarà progresso. Anche chi, in minoranza, crede in idee sbagliate deve essere
lasciato libero perché si deve ridiscutere la propria idea e da ciò nasce l'innovazione. Mill
inoltre tratta il tema dell'\textbf{emancipazione femminile} in alcuni suoi saggi, fra cui 
\textit{``L'asservimento delle donne''}. La società non deve intromettersi nei sentimenti di una
coppia. La donna nella società di Mill vive una condizione simile alla schiavitù, con dei limiti 
nelle professioni, nelle libertà di scelta (prima di vendere un bene, doveva chiederlo al marito). 
Per la donna la migliore condizione era la vedovanza. In un aspetto la situazione era peggiore degli
schiavi: se con essi il legame era evidente, con la donna no, sono infatti educate inconsapevolmente
sin da piccole alla loro inferiorità. Viene fatto passare come qualcosa di conveniente alla donna.
Bisogna cambiare la mentalità della società e dell'istruzione. Secondo Mill, se liberiamo le
donne da questa schiavitù, liberiamo anche la loro intelligenza e quindi ci sarà progresso, 
innovazione.

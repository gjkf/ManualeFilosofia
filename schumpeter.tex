%!TEX ROOT=filosofia.tex

\section{Schumpeter}
Schumpeter è un economista, realista fino a quasi il cinismo. È un \textbf{liberista classico} 
(rifiuta quindi le idee di Keynes). Si incentra sullo viluppo economico, ovvero il prodotto 
dell'innovazione (ovvero applicare qualcosa di nuovo, e non solo innovazione scientifica ma anche 
tecnologica).

\subsection{``Capitalismo, socialismo, democrazia''}
Critica la visione classica della democrazia (quella di Rousseau). \textbf{Il bene comune non esiste}
in quanto esistono solo gli individui, il resto è astrazione. Critica la volontà generale.\\
Nella visione classica si presuppongono individui che politicamente agiscono in modo razionale,
ma non è così, \textbf{gli uomini agiscono in modo irrazionale}, nel proprio campo di 
specializzazione sono razionali ma in politica pochi sono esperti. \textbf{Le paure e le speranze}
guidano il voto.\\ [\baselineskip]
Il mercato capitalistico è preso a modello:
\begin{itemize}
  \item Aziende in concorrenza
  \item Gli imprenditori alla guida
\end{itemize}
Nella democrazia ci sono i \textbf{partiti} e alla loro guida \textbf{i leader} che si organizzano
per prendere il potere attraverso il voto. \textbf{Per prendere voti si fa ``pubblicità''}, si 
guardano gli aspetti irrazionali dell'uomo.

\subsection{Rapporti con Kelsen}
In entrambi i pensatori \textbf{il potere è in mano a pochi}.\\
Il partito o il parlamento non ha il dovere di difendere i valori però. Le elezioni servono solo a
dare il potere in mano a qualcuno. Il problema è \textbf{l'efficienza del governo}, uno deve essere
al potere, il parlamentarismo è troppo lungo.\\
Tra una dittatura e una democrazia c'è solo una differenza: la presenza del voto.

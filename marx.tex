%!TEX ROOT=filosofia.tex

\section{Marx}
Karl Marx è il fondatore del comunismo in senso filosofico nonché un grande conoscitore dell'economia
capitalista. Nel 1844 compone i \textit{'Manoscritti economico-filosofici'}. Nel 1848 pubblica
\textit{'Il manifesto del partito comunista'} in collaborazione con Hengels. Nel 1866 pubblica il
suo scritto principale: \textit{'Il capitale'} (il primo volume).

\subsection{Termini chiave}
\begin{description}
  \item[Ideologia] concezione rovesciata della realtà, presentata come necessaria e materiale. Il
    capitalismo è un'ideologia in quanto crede di essere l'unico e vero sistema economico. Hegel
    credeva che lo stato oggettivasse il bene comune invece è espressione della classe dominante
    che fa i propri interessi.
  \item[Alienazione economica] il capitalismo è alienante nel campo del lavoro
    \begin{description}
      \item[Rispetto al prodotto] il prodotto non è del lavoratore ma del capitalista, il lavoratore
        vede solo una fase della lavorazione.
      \item[Rispetto all'attività] il lavoratore nel capitalismo ripete sempre gli stessi gesti,
        senza creatività, in modo alienante.
      \item[Rispetto al prossimo] il capitalismo induce all'egoismo, riduce i rapporti sociali
        dell'uomo.
    \end{description}
  \item[Alienazione religiosa] gli uomini creano l'alienazione religiosa a causa di quella economica.
    Nella religione cerca una felicità che non può trovare nel lavoro.
\end{description}

\subsection{Critica a Feuerbach}
Feuerbach riteneva che l'uomo fosse natura. Marx gli rimprovera che l'uomo non è solo natura,
\textbf{è anche lavoro}. Si distingue dagli altri esseri viventi per il lavoro. Il lavoro trasforma
il mondo nella storia. Feuerbach è ancora idealista, resta nel campo delle idee, non fa nulla di
pratico.

\subsection{Materialismo storico}
Tutto è mosso da forze economiche. La storia fa i \textbf{modi di produzione}, ovvero l'organizzzione
del lavoro per i beni essenziali.\\
Ci sono due fattori fondamentali della vita sociale e della storia
\begin{description}
  \item[Struttura] base economica della società, fatta da forze produttive (= lavoratori, mezzi di
    produzione) e rapporti di produzione (rapporti di proprietà dei mezzi di produzione). Sono
    rapporti determinati dal sistema economico stesso (esistono le classi sociali e quindi diversi
    interessi economici).
  \item[Sovra-struttura] è la cultura, le idee
\end{description}
La sovra-struttura riflette la struttura (la cultura è legata al lavoro economico). Quanto è stretto
questo rapporto?
\begin{itemize}
  \item La struttura determina la sovra-struttura. Il rapporto è necessario, non c'è liberta per
    l'uomo, l'uomo inevitabilmente in quelle situazioni pensa quelle cose
  \item La struttura condizione la sovra-struttura. La influenza.
\end{itemize}
La storia è sempre stata lotta di classe, la struttura economica genera classi diverse con interessi
diversi. Nel capitalismo la lotta di classe si semplifica: borgesia (dirigenti) e proletariato.
La borghesia è stata una classe rivoluzionaria (la elogia) che ha soppiantato la precedente.
Sviluppandosi il capitalismo si sviluppa il proletariato che si prepara a scalzare la borhesia. Da
qui nasce la \textbf{dialettica della storia}: la borghesia crea la sua antitesi (il proletariato)
e assieme creeranno qualcosa di nuovo (il socialismo).

\subsection{Il capitale}
Nel Capitale, Marx critica il \textbf{feticismo delle merci}. La merce viene presentata come qualcosa
di ovvio, scontato nel mercato capitalistico. In realtà sono prodotti umani. Il valore viene affidato
dall'uomo, non bisogna sottomettersi.
\begin{description}
  \item[Merce] è un qualcosa anche immateriale che deve avere
    \begin{description}
      \item[Valore d'uso] deve servire a qualcosa
      \item[Valore di scambio] deve poter essere scambiato con altre merci (misurato dal denaro)
    \end{description}
\end{description}
L'economista cerca l'origine del valore di scambio di una merce. Deve esserci una cosa comune a
tutte le merci: \textbf{il lavoro}. Nasce così la teoria del \textit{Valore-Lavoro}: il valore
dipende dal lavoro necessario a produrre una merce, è il lavoro sociale, non di un singolo, è
lavoro medio in quanto varia da società a società e con il tempo.\\
Nei sistemi \textbf{pre-capitalisti} l'economia funzionava: Merce, vendita, Denaro, acquisto, 
Merce.\\
Nei sistemi \textbf{capitalisti} l'econimia si basava su: Denaro (capitale), investimento, Merce, 
vendita, Denaro (profitto).\\
Da dove viene fuori il profitto? Il valore deriva dal lavoro, non dallo scambio in quanto è equo,
quindi deve derivare dal lavoro. Un lavoratore produce profitto pari al suo salario (= prezzo del
lavoro, una merce) (= al prezzo minimo della vita). Il salario non è pari al valore che produce.
Un lavoratore lavora $n$ ore per pagarsi il salario (\textbf{lavoro necessario}) e il resto genera
\textbf{plus-lavoro} non retribuito. Quindi genera  \textbf{plus-valore}. Il capitalismo è basato
sullo sfruttamento. Il plus-valore non è ancora profitto. Una parte infatti verrà usata per
investimenti (\textbf{capitale costante}) in quanto c'è concorrenza (i salari son il capitale 
variabile).\\
\textbf{Marx pensa di aver trovato cosa metterà in crisi il capitalismo}. Oltre alla lotta di classe,
si cerca sempre di più di abbassare il salario ma dopo un certo limite non si può andare altrimenti
il lavoratore muore. Si cerca comunque di investire per evitare la concorrenza. L'effetto è quello
di concentrare il capitale in pochissimi uomini (proletarizzazione della borghesia). Avverrà la
\textbf{caduta tendenziale del saggio di profitto}. Il saggio (la percentuale) del profitto rispetto
al capitale tende a diminuire sempre di più.

\subsection{Concezione della rivoluzione e del comunismo}
Marx non era utopista. Non ha dato una chiara descrizione di come sarà il comunismo. \textbf{La
rivoluzione avverrà}, implica l'uso della forza e della violenza ma non è necessario. Il passaggio
può essere graduale, specialmente nei paesi più sviluppati. Ci sono 2 tipi di comunsimo
\begin{description}
  \item[Rozzo] il proletariato prende il potere e lo esercita come classe egemone. Abolisce la
    proprietà privata. Lo stato gestisce l'economia. Il proletariato usa il potere contro la
    borghesia.
  \item[Autentico] stacca completamente dal passato. La proprietà viene completamente abolita. I beni
    non sono più dello stato, vengono autonomamente distribuiti a seconda dei bisogni dell'individuo.
    Con lo stato c'era ancora divisione in classi, senza non c'è rischio. Simil-anarchia. Il
    comunismo autentico è ricco, come se non più del capitalismo.
\end{description}

%!TEX ROOT=filosofia.tex

\section{Marx}
Karl Marx � il fondatore del comunismo in senso filosofico nonch� un grande conoscitore dell'economia
capitalista. Nel 1844 compone i \textit{'Manoscritti economico-filosofici'}. Nel 1848 pubblica
\textit{'Il manifesto del partito comunista'} in collaborazione con Hengels. Nel 1866 pubblica il
suo scritto principale: \textit{'Il capitale'} (il primo volume).

\subsection{Termini chiave}
\begin{description}
  \item[Ideologia] concezione rovesciata della realt�, presentata come necessaria e materiale. Il
    capitalismo � un'ideologia in quanto crede di essere l'unico e vero sistema economico. Hegel
    credeva che lo stato oggettivasse il bene comune invece � espressione della classe dominante
    che fa i propri interessi.
  \item[Alienazione economica] il capitalismo � alienante nel campo del lavoro
    \begin{description}
      \item[Rispetto al prodotto] il prodotto non � del lavoratore ma del capitalista, il lavoratore
        vede solo una fase della lavorazione.
      \item[Rispetto all'attivit�] il lavoratore nel capitalismo ripete sempre gli stessi gesti,
        senza creativit�, in modo alienante.
      \item[Rispetto al prossimo] il capitalismo induce all'egoismo, riduce i rapporti sociali
        dell'uomo.
    \end{description}
  \item[Alienazione religiosa] gli uomini creano l'alienazione religiosa a causa di quella economica.
    Nella religione cerca una felicit� che non pu� trovare nel lavoro.
\end{description}

\subsection{Critica a Feuerbach}
Feuerbach riteneva che l'uomo fosse natura. Marx gli rimprovera che l'uomo non � solo natura,
\textbf{� anche lavoro}. Si distingue dagli altri esseri viventi per il lavoro. Il lavoro trasforma
il mondo nella storia. Feuerbach � ancora idealista, resta nel campo delle idee, non fa nulla di
pratico.

\subsection{Materialismo storico}
Tutto � mosso da forze economiche. La storia fa i \textbf{modi di produzione}, ovvero l'organizzzione
del lavoro per i beni essenziali.\\
Ci sono due fattori fondamentali della vita sociale e della storia
\begin{description}
  \item[Struttura] base economica della societ�, fatta da forze produttive (=lavoratori, mezzi di
    produzione) e rapporti di produzione (rapporti di propriet� dei mezzi di produzione). Sono
    rapporti determinati dal sistema economico stesso (esistono le classi sociali e quindi diversi
    interessi economici).
  \item[Sovra-struttura] � la cultura, le idee
\end{description}

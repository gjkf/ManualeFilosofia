%!TEX ROOT=filosofia.tex

\section{Rawls}
Fu educato da una famiglia religiosa, dopo la seconda guerra mondiale si allontana dalal religione.

\subsection{``Una teoria della giustizia''}
Rawls si propone di creare una \textbf{dottrina etica cognitivistica}, vuole fondare i principi etici
in modo razionale. In questo caso la giustizia è il suo interesse. Vuole creare un'\textbf{etica
normativa} basata sui \textit{principi di giustiza} (riprende Kant e l'imperativo categorico). I
principi di giustizia determinano come dovrebbero essere organizzate le società giuste. \textbf{La
validità dei principi dipende solo dalla razionalità, non dalla loro attuabilità}. Da queste
premesse si deduce che l'etica è pubblica, non privata come Kant.\\
Rawls descrive quello che è il \textbf{primato del giusto} ovvero che la giustizia è la cosa più
importante. Si può fondare razionalmente, non come il concetto di ``bene'' che è completamente
irrazionale e soggettivo.\\ [\baselineskip]
\textbf{Critica l'utilitarismo}: per loro il giusto è ciò che risponde al principio di utilità e
quindi al bene, messo al centro. In realtà non è in grado di produrre un trattamento equo delle
persone. L'equità è da preferire.\\ [\baselineskip]
La sua filosofia è \textbf{neocontrattualista}, alla base c'è un contratto che si deve fondare sul
consenso tra gli individui (per legittimare lo stato). Il contratto \textbf{non} fonda lo stato,
il patto fonda i principi di giustizia.

\subsubsection{La posizione originaria}
Rawls definisce ``posizione originaria'' quello che Hobbes avrebbe definito lo ``stato di natura''.
Gli individui sono razionali, liberi ma non necessariamente altruisti, ognuno guarda solotanto a se
stesso, sono disinteressati degli altri. Devono fare la decisione, l'accordo dietro un \textbf{velo
d'ignoranza} in quanto non sapranno come saranno nel nnuovo stato (ricchi, poveri, operai, nobili,
\ldots). Questo è indispensabile in quanto altrimenti si farebbero norme a proprio favore.\\
Si stabiliranno quindi dei principi di fondo, quelli di giustizia. \textbf{La razionalità è 
stumentale} ad individuare i mezzi adatti per raggiungere uno scopo.

\subsubsection{I principi di giustizia}
Ci sono due principi di giustizia
\begin{description}
  \item[Libertà] Gli individui dovranno godere della massima libertà possibile compatibilmente con
    un'eguale libertà degli altri. È fondamentale, mantiene le libertà personali conciliando quelle
    della società
  \item[Differenza] Le disuguaglianze sono ammesse a due condizioni
    \begin{enumerate}
      \item Devono essere vantaggiose per tutti, specialmente per i più deboli: non è razionale 
        andare a creare una società perfettamente egualitaria in quanto i meriti non sarebbero 
        premiati. Con i premi si incentiva alla produzione, i deboli devono essere aiutati, i più 
        ricchi devono contribuire maggiormente alla società.
      \item Ci sono pari opportunità, le diverse posizioni devono essere accessibili a tutti
    \end{enumerate}
\end{description}
I principi di giustizia dovrebbero essere alla base della costituzione e le leggi ordinarie li 
applicano. Il problema ora è quello di \textbf{dare stabilità} ad una società con diverse visioni
del mondo (queste sono visioni comprensive (come il politeismo dei valori di Mill), è normale in
quanto il bene non può essere definito razionalmente). È necessario qualcosa che dia unità alla
società. Questo è il \textbf{consenso per intersezione}: i principi di giustizia sono comuni a tutti
e questo è una base comune e porta stabilità. Le ragioni pubbliche (ciò su cui si trova un accordo)
si dovrebbero discutere in modo che gli argomenti siano condivisibili da tutti. Lo stato deve essere
neutrale.

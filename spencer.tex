%!TEX ROOT=filosofia.tex

\section{Spencer}
Spencer è un filosofo positivista evoluzionista con una concezione che racchiude sia Darwin che 
Lamarck.

\subsection{Rapporti scienza-religione}
Spencer definisce \textbf{l'Inconoscibile}, ovvero l'inacessibilità della realtà ultima e assoluta.
Questa inaccessibilità mette su di un piano comune la religione e la scienza.\\
In ogni religione la verità ultima è esprimibile come ``l'esistenza del mondo è un mistero che va
interpretato'', però ogni religione fallisce nell'interpretarlo in quanto non ha delle dimostrazioni
logiche. Di conseguenza, \textbf{la religione riconosce che il mistero della natura è imperscrutabile
e ciò che ``Inconoscibile''}.\\
Anche la scienza nella sua ricerca si scontra con delle domande che sono impenetrabili (cosa sia il
tempo, lo spazio, \ldots). \textbf{Le idee scientifiche sono quindi rappresentative di realtà
incomprensibili}.\\
La nostra conoscenza è chiusa entro dei limiti del relativo, il progresso consiste nell'includere
verità sempre maggiori che contenevano le precedenti. \textbf{La verità assoluta non può essere
inclusa in un'altra, quindi è destinata ad essere un mistero}. L' \textbf{Assoluto} quindi è la
forza misteriosa che si manifesta in tutti i fenomeni. Poiché non si può trovare una causa di questa
forza, la religione \textbf{richiamerà il mistero che rappresenta}, la scienza \textbf{estende la
conoscenza fino a questo limite}.\\
Il fenomeno quindi è la manifestazione di questo Inconoscibile. Ogni nozione persistente e immutabile
deiva quindi dall'Inconoscibile, ne è un suo modo di esprimersi. Questa corrispondenza è il
\textbf{realismo trasfigurato}. 

\subsection{Teoria dell'evoluzione}
Qual è il compito della filosofia? La filosofia è \textbf{la conoscenza nel suo più alto grado di 
generalità}. È una conoscenza unificata. Quindi pone come base i principi più ampli a cui la scienza
è giunta. Essi sono
\begin{itemize}
  \item L'indistruttibilità della materia
  \item La continuità del movimento
  \item La persistenza della forza
\end{itemize}
A questi si deve aggiungere la \textbf{legge del ritmo}, ovvero il ciclico alternarsi di fasi acute e
di fasi di caduta.\\
Questi principi richiedono una legge che combini continuamente la materia, essa è 
\textbf{l'evoluzione} secondo cui
\begin{description}
  \item[Si passa dall'incoerente al coerente]
  \item[Si passa dall'omogeneo all'eterogeneo] Ogni organismo prima si sviluppa attraverso la 
    differenziazione delle sue parti, poi si diversificano ulteriormente in tessuti e organi
  \item[Si passa dall'indefinito al definito] Dal vago al preciso
\end{description}
In questo la materia passa da uno stato di dispersione ad uno di integrazione, la forza invece si 
dissipa. L'evoluzione è un passaggio \textbf{necessario} in quanto l'omogeneità è instabile. È 
inoltre necessariamente migliorativo. Anche se per la legge del ritmo ci saranno momenti di
caduta, sono sempre premesse per un'ulteriore evoluzione.

\subsection{Biologia e Psicologia}
La Biologià è lo studio dell'evoluzione dei fenomeni organici. \textbf{La vita è una funzione 
dell'adattamento grazie alla quale organi si formano e si differenziano}. Segue Lamarck secondo cui è
la funzione a creare l'organo, ma segue anche la selezione naturale. Il progresso della vita è
quindi un continuo adattamento all'ambiente.\\ [\baselineskip]
La Psicologia è possibile come scienza autonoma. Ce ne sono di due tipi
\begin{description}
  \item[Ogettiva] che studia i fenomeni psichici
  \item[Soggettiva] che si fonda sull'introspezione
\end{description}
Soltanto la soggettiva può contribuire allo sviluppo del pensiero come adattamento graduale. Spencer
inoltre da anche delle \textbf{nozioni a priori} che sono uniche per l'individuo e non comuni alla
specie.

\subsection{Sociologia e politica}
La sociologia di Spencer è molto diversa da quella di Comte. Infatti per Comte era la massima scienza
quando per Spencer \textbf{deve limitarsi a descrivere lo sviluppo della società umana fino al
presente}. Può studiare le condizioni per lo sviluppo, ma non le mete a cui ambisce che sono invece
definite dalla morale.\\ [\baselineskip]
Spencer si incentra sulla \textbf{difesa delle libertà individuali} e questo lo orienta verso un 
certo individualismo. Lo sviluppo della società dev'essere affidato alla forza spontanea che lo muove
verso il progresso, l'intervento dello Stato rallenta e basta. \textbf{Lo sviluppo sociale è graduale
e inevitabile}. Lo stesso sviluppo sociale ha determinato il passaggio da una cooperazione umana
imposta ad una più libera e spontanea. Questo è il passaggio dal \textbf{regime militare} (prevalenza
del potere statale sugli individui) al \textbf{regime industriale} (fondato sull'indipendenza degli
indivudui). È possibile un terzo regime sociale in cui egoismo e altruismo convivono. 

\subsection{Etica evoluzionistica}
L'etica è biologica, ha per oggetto l'adattamento progressivo dell'uomo alle sue condizioni di vita.
\textbf{L'adattamento non è solo un miglioramento ma è un raggiungimento di maggiore intensità e 
ricchezza della vita}. Il bene si identifica con il piacere.\\
L'uomo singolo agisce per dovere: per un sentimento di obligazzione morale generato da esperienze che
hanno prodotto nell'uomo un sentimento per cui questo sembra più utile per il raggiungimento del 
benessere. \textbf{Il senso di dovere è transitorio} e il progresso e la crescita dell'uomo fanno
scomparire questi obblighi trasformandoli in gesti di altruismo. Questo significa che 
\textbf{altruismo ed egoismo possono essere in perfetto accordo}. 

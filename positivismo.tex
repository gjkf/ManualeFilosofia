%!TEX ROOT=filosofia.tex

\section{Positivismo}
Il positivismo si sviluppa tra la seconda metà dell'800 e gli inizi del '900.
\begin{description}
  \item[Positivo] ciò che è conosciuto in modo diretto, con esperienza. Anche come utile, applicabile
    praticamente.
\end{description}
La scienza è l'unico modo per conoscere la natura ed è un sapere utile per il progresso storico e
tecnico dell'uomo. Viene rifiutata la metafisica romantica, solo i fenomeni sono utili ed esistono.
Viene rifiutata anche la religione, vista come sapere astratto e inutile. Bisogna estendere il
metodo della scienza in tutti i campi, nascono così Psicologia e Sociologia (Comte).

\subsection{Rapporto con l'Illuminismo}
Sia il positivismo che l'Illuminismo hanno in comune
\begin{itemize}
  \item La fiducia nella ragione e nel sapere, visti come mezzo di progresso
  \item Esaltazione della scienza a scapito della metafisica
  \item La visione laica e immanentistica della vita
\end{itemize}
Invece differiscono su altri punti come
\begin{itemize}
  \item Il momento storico è molto diverso e quindi il positivismo manca di una carica polemica
    che era presente nell'Illuminismo (la borghesia ormai si è affermata). Il positivismo è una
    forza riformista consapevolmente anti-rivoluzionaria
  \item La filosofia è vista in modo diverso: gli illuministi la consideravano come una critica
    della scienza, una visione gnoseologica, i positivisiti invece affidano alla filosofia il compito
    di ordinare le scienze e unificarle
  \item La scienza è vista nel positivismo come un sapere assoluto, senza limiti. Nell'Illuminismo
    invece con Hume o Kant erano stati posti dei paletti che la scienza non poteva valicare
\end{itemize}

\subsection{Rapporto con il Romanticismo}
Nonostante ci siano molte differenze tra le due correnti, si possono fare alcune analogie. 
Innanzitutto le differenze principali sono
\begin{itemize}
  \item Il Romanticismo parla in termini di \textit{spirito}, \textit{assoluto}, il Positivismo 
    invece di scienza, Umanità e progresso
  \item Il Romanticismo è espressione di una società pre-industriale, il positivismo è di una
    capitalistica
\end{itemize}
Come somiglianze si può considerare il Positivismo come \textit{romanticismo della scienza}, 
l'esaltazione del sapere positivo.

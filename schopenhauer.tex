%!TEX ROOT=filosofia.tex

\section{Schopenhauer}

Arthur Schopenhauer è un \textbf{romantico critico di Hegel}. Già questo mette in luce una
generale caratterisitca di questo filosofo.\\
La sua opera principale è \textit{Il mondo come volontà e rappresentazione} del 1818. Quest'operà
però porterà successo all'autore solo alla fine degli anni '50 circa.

\subsection{Il mondo come volontà e rappresentazione}
Già nel titolo vengono racchiusi i due termini fondamentali per Schopenhauer: \textbf{volontà} e 
\textbf{rappresentazione}. Già la prima frase dell'opera \textit{'Il mondo è una mia 
rappresentazione'} mette in evidenza il distacco dalla filosofia passata. Se non ci si rende conto
di questa verità, non si può fare filosofia. \textbf{Anche la scienza è una rappresentazione.}.

\begin{description}
  \item[Rappresentazione] conoscenza superficiale delle cose, non l'essenza. Per Kant il 
    \textit{fenomeno}. È da fare la distinzione tra Kant e Schopenhauer: Kant credeva che il fenomeno
    fosse una superficie ma comunque reale, per Schopenhauer invece è un'\textit{illusione}, è una
    maschera
\end{description}

Questo limite posto alla scienza è tipicamente romantico, la scienza infatti non può tutto.\\
La rappresentazione implica
\begin{itemize}
  \item Soggetto che osserva
  \item Oggetto che è osservato
\end{itemize}
\textbf{La filosofia ha l'obiettivo di superare la rappresentazione}, di fare metafisica. È opposto
all'atteggiamento Kantiano della filosofia. Come creare però questa metafisica? Si deve partire dal
corpo. Ognuno di noi ha due modi di conoscere il proprio corpo
\begin{itemize}
  \item Rappresentazione come oggetto fra altri oggetti
  \item Intuizione come il proprio corpo, non quello altrui, della volontà di vivere e delle 
    necessità primarie.
\end{itemize}

\begin{description}
  \item[Volontà] è l'essenza del corpo, è la forza ordinatrice. Tutta la natura ha voglia di vivere,
    ogni cosa. Le forze della natura sono manifestazione di questa voglia di vivere. La volontà
    è unica, eterna, infinita e incausata.\\
    La volontà è anche \textit{mancanza}. Se si desidera qualche cosa non lo si ha, è sofferenza.
\end{description}

La felicità, quindi deriva dall'appagamento del desiderio. La vita è come un pendolo che oscilla tra 
dolore causato dalla volontà e la felicità è solo momentanea, causata dall'appagamento di questa
volontà.\\
Il dato reale dell'esistenza è quindi il dolore. Questo rende la filosofia di Schopenhauer 
pessimistica. Proprio per questo punto è stato considerato come un precursore della \textit{'Scuola
del sospetto'}. Con quest'idea della volontà come causa del dolore, Schopenhauer critica l'idea di
Dio della tradizione: se esistesse Dio, sarebbe un essere crudele in quanto l'uomo diventa
consapevole della sofferenza. 
Quindi \textbf{la religione è un'illusione} per nascondere la realtà.

\subsection{La storia}
Schopenhauer \textbf{critica Hegel} per il suo ottimismo: la visione della storia che vuole essere
razionale, è una maschera. In realtà non è razionale, la vita degli uomini è sempre \textit{volontà
di vivere}. I cambiamenti riguardano solo il fenomeno che Schopenhauer vuole superare. Nella natura
umana non è presente benevolenza, ognuno cerca il proprio vantaggio a discapito degli altri (simile
allo stato di natura di Hobbes). \textbf{Lo stato ha il compito di mantenere l'ordine pubblico e 
garantire la proprietà privata.}

\subsection{L'amore}
L'amore è la \textbf{metafisica dell'anima}. L'idea che sia un sentimento che nobilita l'animo
è una maschera. Non c'è altro che l'istinto sessuale, riproduttivo. \textbf{L'uomo che crede di amare
è in realtà schiavo della volontà}.

\subsection{Vie di liberazione dal dolore}
Ci sono delle modalità per liberarsi dal dolore. Il suicidio non è una di queste in quanto sarebbe
arrendersi alla volontà e volere di non volere. Le vie di liberazione dal dolore sono 3:
\begin{description}
  \item[Arte] è sapere e conoscenza superiore alla scienza, quasi filosofia. L'arte conosce le idee,
    le essenze (una scultura rappresenta un valore generale, non quel particolare soggetto). L'arte
    è contemplazione disinteressata. Il dolore termina, ma è momentanea sospensione.
  \item[Morale] nasce da un sentimento, quello della compasssione, della consapevolezza che la 
    sofferenza è comune. Superiamo l'egoismo ed agiamo in modo disinteressato. Nella morale ci
    sono due aspetti:
    \begin{description}
      \item[Giustizia] non fare del male agli altri (virtù negativa)
      \item[Amore] non come \textit{eros} ma come \textit{agape}, fare il bene degli altri (virtù
        positiva)
    \end{description}
  \item[Ascesi] \textit{noluntas}, negazione radicale della volontà. Negare il desiderio sessuale,
    tutti i bisogni, essere poveri per scelta. Una volta raggiunta l'ascesi, non si sa cosa accade in
    quanto è ineffabile, il linguaggio non può descriverlo. Si raggiunge il nulla dei fenomeni,
    una serenità incomprensibile.
\end{description}

%!TEX ROOT=filosofia.tex

\section{Freud}
Sigmund Freud è il fondatore ed iniziatore della Psicoanalisi. Fornisce una nuova e rivoluzionaria
immagine dell'uomo. Freud stesso in una conferenza definisce la Psicoanalisi come la \textit{``terza
ferita al narcisismo umano.''}. La prima fu quella di \textbf{Copernico} che toglie l'uomo dal centro
dell'universo, la seconda fu \textbf{Darwin} che mette l'uomo al pari degli animali e l'ultima
appunto è la \textbf{Psicoanalisi} che dimostra che noi non siamo padroni di noi stessi.\\
Una cosa importante da tenere a mente è \textbf{concezione positivista} della medicina: tutte le
malattie sono causate da lesioni fisiche e organiche.

\subsection{``Studi sull'isteria''}
L'isteria è una malattia femminile causata da una malformazione/spostamento dell'utero. Freud andò in
Francia da Charcot che notò che quando un paziente aveva una crisi isterica, ipnotizzandolo si 
calmava. Con questo Freud ipotizzò che non c'erano solo cause organiche ma anche una qualche 
esperienza dolorosa dimenticata. Così Freud torna a Vienna e si mette in proprio assieme a Breuer.
I pazienti di Freud sono benestanti e acculturati.

\subsubsection{Associazioni libere}
Il metodo delle associazioni libere consiste nel \textbf{associare liberamente idee e pensieri ai
sintomi, anche senza apparente collegamento.} Nella vita psichica, niente è casuale e questo 
permetteva di trovare la causa corretta dei sintomi. Anna O. (una delle pazienti fondamentali degli
studi di Freud) aveva visto il cane della governante bere dal suo bicchiere e questo è stato
un trauma per il suo inconscio di bambina. Da qui nasce l'idrofobia durante una crisi. Con le 
associazioni libere si scarica la tensione emotiva e avviene un processo di \textbf{catarsi} che 
porta alla guarigione.\\
L'ipnosi blocca i sintomi ma non aiuta a trovare le cause. Così Freud abbandona questa pratica tranne
estremi casi. Questo comincia ad allontanarlo da Breuer.

\subsubsection{Transfert}
È il legame emotivo che si viene a creare tra il medico e il paziente. Per Freud il medico deve 
accettare questo legame. Breuer invece lo criticava giudicandolo un atteggiamento positivista. 
Bisogna avere fiducia assoluta nel medico.

\subsubsection{Sfera sessuale}
Per Freud i traumi più gravi li si hanno avuti in età infantile. Anche violenze sessuali temute o
immaginate. Breuer invece rifiuta questa sfera.

\subsection{``L'interpretazione dei sogni''}
Pubblicato nel 1900, questo famosissimo libro è stato composto da Freud per due motivi
\begin{enumerate}
  \item Nel 1896 comincia a sognare il padre che era morto
  \item Molti dei suoi pazienti parlavano dei sogni che avevano fatto
\end{enumerate}
I positivisti credevano che il sogno fosse irrilevante, una scarica della tensione quotidiana.\\
Freud voleva indagare il sogno \textbf{scientificamente}. Il sogno parla con un altro mondo:
\textbf{l'inconscio}. Quando dormiamo la coscienza è affievolita e il sogno permette di arrivare
all'inconscio. Per Freud il sogno è ``appagamento onirico di un desiderio inconscio''. Il desiderio
spesso è incofessabile, anche a noi stessi, anche gli incubi hanno un desiderio.

\subsubsection{Lavoro onirico}
I desideri non si esprimono in mododiretto, durante il sonno l'inconscio si manifesta ma la coscienza
maschera il desiderio perché spesso è inaccettabile. Questo è il \textbf{lavoro onirico}: mascherare
un desiderio. Ci sono due contenuti del sogno
\begin{description}
  \item[Latente] è il vero significato
  \item[Manifesto] quello che osserviamo, ciò che la coscienza usa per mascherare
\end{description} 
Il lavoro onirico avviene in due fasi:
\begin{description}
  \item[Condensazione] Nel sogno tutto è fuso, perciò è caotico, strano, confuso
  \item[Spostamento] Il sogno pone l'attenzione su qualcosa non rilevante nel sogno
\end{description}
Lo psicanalista va alla ricerca del contenuto latente tramite le associazioni libere.\\
[\baselineskip]
I sogni più importanti sono quelli di \textbf{morte di persone care}. Ci sono due tipi di sogni di
morte
\begin{description}
  \item[Senza provare dolore] Hanno significati minori, poco importanti
  \item[Provando dolore, fino a piangere] Dietro questi sogni c'è il desiderio, inaccettabile, della
    morte dellapersona. Il desiderio però può anche essere stato provato una volta nella vita,
    non necessariamente in questo momento. Inoltre il bambino è egoista (il bambino ha avuto 
    l'esperienza traumatica) e conosce solo i propri bisogni. Infine la morte per un bambino non è
    la morte per un adulto, è allontanamento
\end{description}
I sogni di morte dei genitori sono estremamente importanti. Il \textbf{bambino} fin da piccolo ha
\textbf{pulsioni sessuali} collegate a ciò che prova. Verso i 4 anni rivolge queste pulsioni verso
il genitore di sesso opposto e vede il genitore dello stesso sesso come ``rivale in amore''. Quindi
da qui nasce il desiderio di allontanamento del genitore, e quindi la sua morte.

\subsection{``3 saggi sulla teoria sessuale''}
Precedentemente si credeva che il bambino avesse una sessualità diversa dall'anziano, per Freud 
invece la sessualità muta durante le diverse fasi della vita. La sessualità non si identifica con la
genitalità, è una pulsione di piacere, una \textbf{libido}. Si possono distinguere le varie fasi
\begin{description}
  \item[Fase orale] Si sviuppa dalla bocca, il bambino succhia il latte e prova piacere a farlo
  \item[Fase anale] Dopo lo svezzamento (1 o 2 anni di età) controlla le funzioni fisiologiche e
    prende il controllo del suo corpo e aumenta l'auto considerazione
  \item[Fase edipica] I genitori devono insegnare che l'incesto è vietato, i genitori devono dire di
    ``NO!''. Da qui comincia l'educazione, la ``civilizzazione'' che prevede regole e divieti. Il
    genitore deve trovare il giusto mezzo. Se l'educazione manca, il bambino potrebbe avere problemi
    di personalità. Senza educazione potrebbe sviluppare perversioni (cercare piacere in modi non
    ammessi dalla morale). Se l'educazione è troppo rigida, diventerà una personalità regredita che
    quando prova piacere si sente in colpa
\end{description}
Freud definisca anche la \textbf{sublimazione} ovvero le pulsioni vengono deviate dalla libido verso
altre mete socialmente apprezzate come il gioco ad esempio.

\subsection{``Psicopatologia della vita quotidiana''}
In questo libro Freud si interessa a fenomeni che prima di allora non erano mai stati presi in 
considerazione, non di malattie vere e proprie ma di \textbf{atti mancati} come ad esempio i
\textit{lapsus}. Questi non sono casuali, sono derivati dalla stanchezza che causa l'indebolimento
della coscienza e così l'inconscio può manifestarsi apertamente. Dato che questo non riguarda solo
una ristretta categoria di persone ma tutti quanti, \textbf{la psicoanalisi mette in discussione la
linea netta tra salute e malattia}. 

\subsection{Scritti e successo dopo il 1920}
Prima della guerra, Freud ebbe un grosso successo e venne chiamato anche negli stati uniti a tenere
conferenze. Fondò anche la \textbf{Società Psicoanalitica Internazionale}.\\ [\baselineskip]
Nel 1920 scrive \textbf{``Al di là del principio di piacere''} in cui rivede il concetto di libido.
Infatti ora Freud identifica due momenti
\begin{description}
  \item[Eros] La libido di cui parlava precedentemente
  \item[Thanatos] La pulsione di morte, di disgregazione (nata probabilmente dopo la guerra 
    sottoforma di pessimismo ripreso da Schopenhauer e Nietsche)
\end{description}
Freud e Einstein si scrivevano delle lettere. L'ultimo scrisse al dottore se mai nella storia ci 
potesse essere una possibilità della cessazione di ogni guerra. Freud disse di no, in quanto la 
guerra è l'espressione principale di Thanatos. \textbf{La visione della storia è pessimistica}.\\
[\baselineskip]
Nel 1922 scrisse \textbf{``L'Io e l'Es''} in cui descrive la seconda topica. La prima topica era
\begin{description}
  \item[Inconscio] Del rimosso
  \item[Preconscio] Facile da raggiungere
  \item[Coscienza] Svegli
\end{description}
La seconda invece
\begin{description}
  \item[Es] La pulsione. L'Es è un \textit{calderone ribollente} di pulsioni di origine organica,
    \textbf{irrazionale, a-morale}. Non tengono conto della realtà e richiedono un \textbf{immediato
    appagamento}. È tipica del bambino
  \item[Io] Un compromesso instabile tra Es e Super-Io
  \item[Super-Io] I valori morali. Dopo il complesso edipico, è opposto all'Es, sono le regole della
    vita sociale che controllano le pulsioni dell'Es. Mettere sotto controllo le pulsioni però porta
    una certa sofferenza e dolore interiore. Il processo di apprendimento si dimentica e nell'età
    adulta le regole morali sembrano innate
\end{description}

\subsection{``Il disagio della civiltà''}
La civiltà implica necessariamente un disagio (è simile all'alienazione marxista) perché può esistere
soltanto se ci sono regole e quindi divieti che causano \textbf{dolore e frustrazione}. La civiltà
ha il vantaggio che rende la vita degli uomini sicura, se non ci fossero limiti sarebbe come lo stato
di natura di Hobbes, la guerra di tutti contro tutti. Freud non crede ci sarà una società felice. Il 
progresso che complica la civiltà e la società. Esso genera nuove regole e quindi nuovo dolore.\\
In URSS sta venendo fuori una società egalitaria e libera, Freud non ci crede molto in quanto una
società implica delle regole e quindi dolore.\\
Le \textbf{valvole di sfogo} sono la sublimazione delle pulsioni (come il superamento del complesso
edipico o lo spostamento).

\subsection{``L'avvenire di un'illusione''}
La religione è un'illusione (non solo come idea falsa, ma anche come appagamento di un desiderio 
inconscio, un po' come il sogno). Il desiderio è \textbf{di aiuto e protezione}, è un desiderio
infantile, come il bambino che vuole il padre. \textbf{Dio è la proiezione del pade}. La religione
controlla la \textbf{pulsione di Thanatos} nella società. La maggior parte degli uomini ha bisogno
della religione.\\
\textbf{Possono gli uomini fare a meno di Dio?} Forse in un futuro quando la scientificità prende il
sopravvento la religione perderà importanza. Comunque una buona parte degli uomini avrà sempre 
bisogno della religione (come Nietzsche e la morale degli schiavi), credono in ciò che desiderano e 
di cui hanno bisogno.

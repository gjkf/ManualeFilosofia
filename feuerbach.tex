%!TEX ROOT=filosofia.tex

\section{Feuerbach}
Ludwig Feuerbach è il fondatore del \textbf{materialismo filosofico ottocentesco}, nonché anche
esponente della sinistra Hegeliana. GLi scritti fondamentali sono \textit{'Critica della filosofia
Hegeliana'}, \textit{'L'essenza del cristianesimo'} e  \textit{'L'essenza della religione'}.

\subsection{Il rovesciamento dei rapporti di predicazione}
Nel criticare Hegel, Feuerbach critica il rapporto tra concreto e astratto. La natura, dice 
Feuerbach, è materia, natura, non spirito assoluto. Un pensiero simile lo rivolge alla 
\textbf{religione}. La religione parte da un'astrazione (Dio) da cui fa nascere la natura e tutte le 
cose. \textbf{Dio è solo una proiezione degli uomini}. Quindi si rovescia ciò che è scritto nella 
Bibbia. A partire dalla propria visione della vita, gli uomini creano una divinità. Dio ha le
capacità umane elevate alla perfezione.\\
Se si vuole conoscere un popolo si deve conoscere la sua religione perché in essa si esprime la
cultura e il pensiero del popolo. La \textbf{religione è} quindi \textbf{autocoscienza}, indiretta
e capovolta ovvero non si è consapevoli di non conoscere il vero (si crede di conoscere Dio come
vera entità ma non è così!).\\
Se si chiede ad un fedele cosa crede delle altre religioni, dirà che sono invenzioni umane. Feuerbach
fa questo per tutte le religioni.\\
Essere atei non significa negare ogni valore alla religione. Essa infatti è la prima forma di
autocoscienza che è indispensabile.\\
La religione e la filosofia conoscono la stessa cosa per Hegel l'assoluto, per Feuerbach l'uomo.
\begin{description}
  \item[Alienzione religiosa] essere qualcosa che non si è, non riuscire a realizzarsi come uomini,
    l'uomo proietta in Dio sè stesso all'infinito quindi l'uomo punta ad essere Dio e disprezza la
    sua finitezza. \textbf{La religione è pericolosa.}
  \item[Rovesciamento dei rapporti di predicazione] `Rimettere la filosofia con i piedi per terra.'
    Quello che nella religione è il predicato, deve diventare soggetto. (Nella religione `Dio è 
    amore', nella filosofia `L'amore è qualcosa di divino')
\end{description}

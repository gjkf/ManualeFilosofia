%!TEX ROOT=filosofia.tex

\section{Feuerbach}
Ludwig Feuerbach � il fondatore del \textbf{materialismo filosofico ottocentesco}, nonch� anche
esponente della sinistra Hegeliana. GLi scritti fondamentali sono \textit{'Critica della filosofia
Hegeliana'}, \textit{'L'essenza del cristianesimo'} e  \textit{'L'essenza della religione'}.

\subsection{Il rovesciamento dei rapporti di predicazione}
Nel criticare Hegel, Feuerbach critica il rapporto tra concreto e astratto. La natura, dice 
Feuerbach, � materia, natura, non spirito assoluto. Un pensiero simile lo rivolge alla 
\textbf{religione}. La religione parte da un'astrazione (Dio) da cui fa nascere la natura e tutte le 
cose. \textbf{Dio � solo una proiezione degli uomini}. Quindi si rovescia ci� che � scritto nella 
Bibbia. A partire dalla propria visione della vita, gli uomini creano una divinit�. Dio ha le
capacit� umane elevate alla perfezione.\\
Se si vuole conoscere un popolo si deve conoscere la sua religione perch� in essa si esprime la
cultura e il pensiero del popolo. La \textbf{religione �} quindi \textbf{autocoscienza}, indiretta
e capovolta ovvero non si � consapevoli di non conoscere il vero (si crede di conoscere Dio come
vera entit� ma non � cos�!).\\
Se si chiede ad un fedele cosa crede delle altre religioni, dir� che sono invenzioni umane. Feuerbach
fa questo per tutte le religioni.\\
Essere atei non significa negare ogni valore alla religione. Essa infatti � la prima forma di
autocoscienza che � indispensabile.\\
La religione e la filosofia conoscono la stessa cosa per Hegel l'assoluto, per Feuerbach l'uomo.
\begin{description}
  \item[Alienzione religiosa] essere qualcosa che non si �, non riuscire a realizzarsi come uomini,
    l'uomo proietta in Dio s� stesso all'infinito quindi l'uomo punta ad essere Dio e disprezza la
    sua finitezza. \textbf{La religione � pericolosa.}
  \item[Rovesciamento dei rapporti di predicazione] `Rimettere la filosofia con i piedi per terra.'
    Quello che nella religione � il predicato, deve diventare soggetto. (Nella religione `Dio � 
    amore', nella filosofia `L'amore � qualcosa di divino')
\end{description}

%!TEX ROOT=filosofia.tex

\section{Rosselli}
È interventista e ha una visione socialista della democrazia ma diversa da Gramsci. Nel '30 fonda
``Giustizia e Libertà'' un movimento anti fascista che diventerà il Partito d'Azione durante la 
resistenza.\\
È \textbf{contro Marx} in quanto crede abbia fallito. La visione marxista è una visione complessiva
della storia, una teoria scientifica che però non tiente in conto l'uomo e le sue libertà. È una 
visione deterministica. Le previsioni di Marx non si sono avverate (come diceva Bernstein) e i
riformisti sbagliano a voler aggiustare il marxismo.

\subsection{``Socialismo liberale''}
Normalmente sono in contrasto: il liberalismo protegge l'individuo, il socialismo la classe. Per 
Rosselli però possono convivere: \textbf{il socialismo deriva da uno sviluppo del liberalismo}. Il
socialismo vuole liberare gli uomini dal bisogno (povertà, \ldots) in quanto sono ostacoli alla
libertà dell'uomo. Ciò non vuol dire negare le altre libertà, ma aumentarle. L'URSS non è socialismo
liberale in quanto lì la borghesia è stata distrutta. Invece va mantenuta e ampliata. Bisogna 
estendere la democrazia oltre la politica, nelle fabbriche. L'economia quindi diventa mista (statale-
privata).

%!TEX ROOT=filosofia.tex

\section{Maritain}
Maritain fu educato da laico positivista. Segue Bergson e il suo spiritualismo per poi convertirsi al
cristianesimo cattolico.

\subsection{``Umanesimo integrale''}
La sua filosofia può essere definita \textbf{neoscolastica} in quanto riprende S.~Tommaso. Ha una
visione critica di tutta la filosofia moderna che si è allontanata da Dio. Da Cartesio non è più Dio,
l'Essere al centro, ma l'uomo. Questo implica una visione parziale dell'uomo. Maritain prevede un
\textbf{umanesimo integrale}, ovvero una visione complessiva e generale dell'uomo. Anche nella
politica è così, da Machiavelli che ha separato etica e politica.\\ [\baselineskip]
Lo scopo della politica è \textbf{il bene comune}. L'idea che lo stato è un potere sovrano porta
all'assolutismo. \textbf{L'unico sovrano è Dio}. \textbf{Assoluto è solo Dio}. Seguendo S.~Tommaso
crede che Dio crei tutte le cose e gli uomini con certi criteri, dà alla natura un ordine, un
``diritto naturale''. Il potere è del popolo che elegge i rappresentanti che governano per il bene
di tutti. Al di sopra del popolo c'è solo Dio e il diritto naturale (nega Rousseau). \textbf{Una 
legge ha valore solo se è coerente con il diritto naturale}.\\
La dottrina liberale e socialista sono prodotti dell'antropocentrismo.
\begin{description}
  \item[Liberali] mettono al centro l'individuo. L'uomo però non è solo individuo, ha anche una 
    socialità che i liberali negano
  \item[Socialisti] sono ``eresia crisitiana''. Viene sottolineata la natura sociale dell'uomo e 
    quindi c'è qualcosa di cristiano (l'attenzione ai deboli, \ldots) ma il marxismo è ateo
\end{description}
Entrambe le visioni sono parziali (``frammenti di verità''). Maritain le vuole unire. L'uomo è sì
inserito in una società ma ha anche una parte spirituale.

\subsection{Democrazia}
Deve ridare unità alle verità frammentarie dell'epoca moderna. Nel Medioevo la religione teneva unito
il popolo (era una \textbf{società sacrale}). Non è più possibile quell'unità in quanto non c'è più
unità nella fede.\\
La democrazia è \textbf{un'insieme di regole che definiscono i rapporti con altri individui}. Deve
esserci rispetto della persona, libertà personali, di associazione, di religione (il cattolicesimo
non è necessario, le altre alternative (guerra di religione) sono peggio, non si può tornare al
Medioevo). Questi principi si ricavano dai testi evangelici e sono compatibili con il cristianesimo.
\textbf{L'obiettivo dei rappresentanti del popolo} è il bene comune, la felicità terrena. Del 
Medioevo Maritain vuole riprendere e migliorare l'\textbf{autonomia della politica} dalla religione.
Sulle regole si può trovare il consenso di fedi diverse, anche atei. Per questo la democrazia dà
unità al mondo moderno. Daranno il consenso per motivi diversi ma lo daranno.

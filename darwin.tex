%!TEX ROOT=filosofia.tex

\section{Darwin}
Darwin è il padre dell'evoluzionismo biologico. Era di famiglia benesatante con un forte impatto
scientifico (erano molti naturalisti, medici). La sua massima opera è \textit{``L'origine delle 
specie''}.\\
Nel 1831 si imbarca in un viaggio finanziato dal governo inglese che aveva l'obiettivo di esplorare
il Sud America e le isole circostanti. Il viaggio duro \textbf{5 anni} nei quali Darwin raccolse 
molti dati ed informazioni su cui poi lavorerà tutta la vita.

\subsection{L'origine delle Specie}
Fu influenzato dallo scritto di Malthus \textit{``Saggio sul principio di popolazione''} in cui 
esprimeva l'idea di uno squilibrio tra risorse e popolazione. Questo squilibrio porterà ad una lotta
per la sopravvivenza.\\
L'evoluzione è causata dalle \textbf{piccole variazioni} che sono i cambiamenti naturali che si 
vedono tra genitori e figli. Alle Galapagos Darwin ebbe il modo di criticare Lamarck in quanto nello
stesso ambiente, si potevano vedere specie fondamentalmente diverse. Le piccole variazioni sono sia 
favorevoli che sfavorevoli e sopratutto \textbf{non si ereditano i caratteri acquisiti}. L'ambiente
non ha una funzione di selezione, non è causa dell'evoluzione. C'è invece una \textbf{lotta per la
sopravvivenza} che invece seleziona le specie. La lotta è sia tra specie diverse che all'interno 
della stessa specie.\\ [\baselineskip]
Ricevette grandi elogi ma anche forti critiche in quanto si toccavano questioni delicate e Darwin
non sapeva spiegare le piccole variazioni e come mai si presentassero. Alcune critiche furono mosse
da Kelvin che gli rimproverava che la Terra era troppo giovane perché l'evoluzione potesse essere
credibile (questa teoria era sbagliata). Darwin la prende molto sul serio e modifica leggermente la 
sua idea dicendo che anche l'ambiente può accelerare il processo.

\subsection{Il rapporto uomo-animale}
Non ne parla mai nell'Origine delle Specie, ma in altri scritti più tardi. L'uomo è un essere 
naturale che è sottoposto alle stesse leggi degli animali. L'uomo non differisce dagli animali per
qualità, ma per grado. Infatti anche gli animali hanno una certa intelligenza, solo di grado 
inferiore all'uomo. Lyell, Wallace e altri credoon che la \textbf{morale} differisca l'uomo dagli 
animali in quanto non è spiegabile nell'evidenza biologica. Darwin invece crede che sia solo una
strategia di sopravvivenza. Darwin non è finalista, non crede ci sia un Dio buono in un mondo così
sofferente.

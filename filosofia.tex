\documentclass[usenames,a4paper,dvipsnames]{article}
\usepackage[italian]{babel}
\usepackage[latin1]{inputenc}
\usepackage{bm}
\usepackage{amsmath, amsfonts, amssymb, mathtools, calrsfs}
\usepackage[pdftex,dvipsnames]{xcolor}
\usepackage[colorinlistoftodos,prependcaption,textsize=tiny]{todonotes}
\usepackage{xtab, booktabs, array}
\usepackage[labelformat=empty]{caption}
\usepackage{tikz}
\usetikzlibrary{arrows,decorations,calc,intersections,shapes.geometric}
\usepackage{parskip} % To avoid indentation
\usepackage{textcomp} % To have \textcelsius and other symbols
\usepackage{multirow} % For nicer tables with split rows/columns
\usepackage{multicol}
\usepackage{cancel} % Cacelled equations and simplifications
\usepackage{hyperref}
\hypersetup{
    colorlinks,
    citecolor=black,
    filecolor=black,
    urlcolor=black,
    linkcolor=blue
}

\usepackage[showframe=false, top=2cm, bottom=2.5cm, left=2.5cm, right=2.5cm]{geometry}

\usepackage[all]{hypcap}
\usepackage{xargs}
% Some useful TODO commands
\newcommandx{\unsure}[2][1=]{\todo[linecolor=red,backgroundcolor=red!25,bordercolor=red,#1]{#2}}
\newcommandx{\change}[2][1=]{\todo[linecolor=blue,backgroundcolor=blue!25,bordercolor=blue,#1]{#2}}
\newcommandx{\info}[2][1=]{\todo[linecolor=OliveGreen,backgroundcolor=OliveGreen!25, bordercolor=OliveGreen,#1]{#2}}
\newcommandx{\improvement}[2][1=]{\todo[linecolor=Plum,backgroundcolor=Plum!25, bordercolor=Plum,#1]{#2}}
\newcommandx{\thiswillnotshow}[2][1=]{\todo[disable,#1]{#2}}
\setlength{\marginparwidth}{2cm}

% For an older but clearer root. Still \oldsqrt is valid
\usepackage{letltxmacro}
\makeatletter
\let\oldr@@t\r@@t
\def\r@@t#1#2{%
  \setbox0=\hbox{$\oldr@@t#1{#2\,}$}\dimen0=\ht0
  \advance\dimen0-0.2\ht0
  \setbox2=\hbox{\vrule height\ht0 depth -\dimen0}%
  {\box0\lower0.4pt\box2}}
\LetLtxMacro{\oldsqrt}{\sqrt}
\renewcommand*{\sqrt}[2][\ ]{\oldsqrt[#1]{#2} }
\makeatother

% Matrix spacing
\makeatletter
\renewcommand*\env@matrix[1][\arraystretch]{%
  \edef\arraystretch{#1}%
  \hskip -\arraycolsep
  \let\@ifnextchar\new@ifnextchar
  \array{*\c@MaxMatrixCols c}}
\makeatother

\newcommand\twodigits[1]{%
  \ifnum#1<10 0\number#1 \else #1\fi
}
\usepackage[yyyymmdd]{datetime}
\renewcommand{\dateseparator}{-}
\usepackage{fancyhdr}
\pagestyle{fancy}
\fancyhead{} % clear all header fields
\renewcommand{\headrulewidth}{0pt} % no line in header area
\fancyfoot{} % clear all footer fields
\fancyfoot[C,CO]{\thepage}% page number in "outer" position of footer line
\fancyfoot[R,RO]{Copyright \copyright 2017--\the\year$\,$Cossu Davide
} % other info in "inner" position of footer line
\fancyfoot[L,LO]{Version 0.0.1 \today
}
\DeclarePairedDelimiter\norm{\lVert}{\rVert} % ||v||  
\DeclarePairedDelimiter\abs{\lvert}{\rvert} % |v|
\newcommand\markangle[7][red]{% [color] origin A B radius radiusmark mark
  % fill red circle
  \begin{scope}
    \path[clip] (#2) -- (#3) -- (#4);
    \fill[color=#1,fill opacity=0.5,draw=#1,name path=circle]
    (#2) circle (#5);
  \end{scope}
  % middle calculation
  \path[name path=line one] (#2) -- (#3);
  \path[name path=line two] (#2) -- (#4);
  \path[%
    name intersections={of=line one and circle, by={inter one}},
    name intersections={of=line two and circle, by={inter two}}
  ] (inter one) -- (inter two) coordinate[pos=.5] (middle);
  % put mark
  \node at ($(#2)!#6!(middle)$) {#7};
}
\def\mathcolor#1#{\@mathcolor{#1}}
\def\@mathcolor#1#2#3{%
      \protect\leavevmode
      \begingroup
      \color#1{#2}#3%
      \endgroup
}
% For better visual in tables
\renewcommand*{\arraystretch}{2}
% To center with m{}
\newcolumntype{M}[1]{>{\centering\arraybackslash}m{#1}}
\newcommand{\divisor}{\rule{8.7cm}{0.4pt}}

\begin{document}
{
\hypersetup{linkcolor=black}
\tableofcontents
}

%!TEX ROOT=filosofia.tex

\section{Schopenhauer}

Arthur Schopenhauer è un \textbf{romantico critico di Hegel}. Già questo mette in luce una
generale caratterisitca di questo filosofo.\\
La sua opera principale è \textit{Il mondo come volontà e rappresentazione} del 1818. Quest'operà
però porterà successo all'autore solo alla fine degli anni '50 circa.

\subsection{``Il mondo come volontà e rappresentazione''}
Già nel titolo vengono racchiusi i due termini fondamentali per Schopenhauer: \textbf{volontà} e 
\textbf{rappresentazione}. Già la prima frase dell'opera \textit{'Il mondo è una mia 
rappresentazione'} mette in evidenza il distacco dalla filosofia passata. Se non ci si rende conto
di questa verità, non si può fare filosofia. \textbf{Anche la scienza è una rappresentazione.}.

\begin{description}
  \item[Rappresentazione] conoscenza superficiale delle cose, non l'essenza. Per Kant il 
    \textit{fenomeno}. È da fare la distinzione tra Kant e Schopenhauer: Kant credeva che il fenomeno
    fosse una superficie ma comunque reale, per Schopenhauer invece è un'\textit{illusione}, è una
    maschera
\end{description}

Questo limite posto alla scienza è tipicamente romantico, la scienza infatti non può tutto.\\
La rappresentazione implica
\begin{itemize}
  \item Soggetto che osserva
  \item Oggetto che è osservato
\end{itemize}
\textbf{La filosofia ha l'obiettivo di superare la rappresentazione}, di fare metafisica. È opposto
all'atteggiamento Kantiano della filosofia. Come creare però questa metafisica? Si deve partire dal
corpo. Ognuno di noi ha due modi di conoscere il proprio corpo
\begin{itemize}
  \item Rappresentazione come oggetto fra altri oggetti
  \item Intuizione come il proprio corpo, non quello altrui, della volontà di vivere e delle 
    necessità primarie.
\end{itemize}

\begin{description}
  \item[Volontà] è l'essenza del corpo, è la forza ordinatrice. Tutta la natura ha voglia di vivere,
    ogni cosa. Le forze della natura sono manifestazione di questa voglia di vivere. La volontà
    è unica, eterna, infinita e incausata.\\
    La volontà è anche \textit{mancanza}. Se si desidera qualche cosa non lo si ha, è sofferenza.
\end{description}

La felicità, quindi deriva dall'appagamento del desiderio. La vita è come un pendolo che oscilla tra 
dolore causato dalla volontà e la felicità è solo momentanea, causata dall'appagamento di questa
volontà.\\
Il dato reale dell'esistenza è quindi il dolore. Questo rende la filosofia di Schopenhauer 
pessimistica. Proprio per questo punto è stato considerato come un precursore della \textit{'Scuola
del sospetto'}. Con quest'idea della volontà come causa del dolore, Schopenhauer critica l'idea di
Dio della tradizione: se esistesse Dio, sarebbe un essere crudele in quanto l'uomo diventa
consapevole della sofferenza. 
Quindi \textbf{la religione è un'illusione} per nascondere la realtà.

\subsection{La storia}
Schopenhauer \textbf{critica Hegel} per il suo ottimismo: la visione della storia che vuole essere
razionale, è una maschera. In realtà non è razionale, la vita degli uomini è sempre \textit{volontà
di vivere}. I cambiamenti riguardano solo il fenomeno che Schopenhauer vuole superare. Nella natura
umana non è presente benevolenza, ognuno cerca il proprio vantaggio a discapito degli altri (simile
allo stato di natura di Hobbes). \textbf{Lo stato ha il compito di mantenere l'ordine pubblico e 
garantire la proprietà privata.}

\subsection{L'amore}
L'amore è la \textbf{metafisica dell'anima}. L'idea che sia un sentimento che nobilita l'animo
è una maschera. Non c'è altro che l'istinto sessuale, riproduttivo. \textbf{L'uomo che crede di amare
è in realtà schiavo della volontà}.

\subsection{Vie di liberazione dal dolore}
Ci sono delle modalità per liberarsi dal dolore. Il suicidio non è una di queste in quanto sarebbe
arrendersi alla volontà e volere di non volere. Le vie di liberazione dal dolore sono 3:
\begin{description}
  \item[Arte] è sapere e conoscenza superiore alla scienza, quasi filosofia. L'arte conosce le idee,
    le essenze (una scultura rappresenta un valore generale, non quel particolare soggetto). L'arte
    è contemplazione disinteressata. Il dolore termina, ma è momentanea sospensione.
  \item[Morale] nasce da un sentimento, quello della compasssione, della consapevolezza che la 
    sofferenza è comune. Superiamo l'egoismo ed agiamo in modo disinteressato. Nella morale ci
    sono due aspetti:
    \begin{description}
      \item[Giustizia] non fare del male agli altri (virtù negativa)
      \item[Amore] non come \textit{eros} ma come \textit{agape}, fare il bene degli altri (virtù
        positiva)
    \end{description}
  \item[Ascesi] \textit{noluntas}, negazione radicale della volontà. Negare il desiderio sessuale,
    tutti i bisogni, essere poveri per scelta. Una volta raggiunta l'ascesi, non si sa cosa accade in
    quanto è ineffabile, il linguaggio non può descriverlo. Si raggiunge il nulla dei fenomeni,
    una serenità incomprensibile.
\end{description}

%!TEX ROOT=filosofia.tex

\section{Kierkegaard}
Soren Aabye Kierkegaard � un filosofo \textbf{critico di Hegel}. Le sue opere principali sono
\textit{Aut-aut} e \textit{Timore e tremore}. Scriveva per difendere il cristianesimo dagli attacchi,
era critico dei luterani danesi.

\subsection{La categoria del singolo}
In Kierkegaard � fondamentale la categoria del singolo. \textbf{Quello che conta ed � reale � il 
singolo individuo, il popolo, la nazione sono tutte astrazioni.} Il valore della vita dipende 
dall'originalit� del singolo individuo. Rifiuta perci� l'idealismo e il sistemismo: racchiudere in u
unico sistema tutta la realt� � impossibile e insensato.

\subsection{La possibilit�}
Centrale in Kierkegaard � il tema della scelta. La scelta � un \textbf{salto nel vuoto}, la scelta
ci mette di fronte al nulla. Le possibilit� non scelte resteranno nel nulla. Ci sono 3 possibilit�
di fondo, o stadi dell'esistenza
\begin{description}
  \item[Esistenza estetica] Don Giovanni � preso a riferimento. La vita � dedicata al piacere e al
    godimento. Si vive nell'attimo, si vuole evitare la ripetizione. Il godimento � fisico (sessuale)
    e psicologico (della conquista del potere). � destinata alla disperazione in quanto non ha una
    continuit� e un'identit�.
  \item[Esistenza Etica] Giudice Guglielmo � il personaggio. � una vita guidata da valori morali ed
    etici. � marito (continuit�), padre, ha un lavoro onesto. Ha una storia e una personalit�. 
    Giunger� alla tristezza in quando adeguandosi ai valori morali, si uniformer� alla comunit�,
    rifiutando la singolarit�. Si pentir� dei suoi errori.
  \item[Esistenza Religiosa] Abramo � il riferimento. Deve scegliere se sacrificare Isacco, l'ordine
    di Dio � contro la morale, � una scelta irrazionale. La fede quindi � abbandonarsi a Dio senza
    sicurezze e garanzie. � una scelta individuale. Agamennone deve sacrificare Ifigenia. La 
    situazione � diversa perch� ne parla con altri e la scelta � comprensibile (sacrificare la figlia
    per un bene maggiore).
\end{description}
Questi tre stadi non sono compatibili fra di loro. Sono mutualmente esclusivi.

\subsection{L'angoscia}
L'angoscai � la percezione del nulla prima di una scelta. Non � paura. Quando scegliamo siamo di 
fronte al nulla e non ci sono garanzie che la scelta sia giusta. Questa libert� pu� portare al 
peccato.

%!TEX ROOT=filosofia.tex

\section{Correnti post-Hegeliane}
Gli allievi di Hegel si dividono in due correnti: la \textbf{Sinistra} e la \textbf{Destra} 
hegeliana. Principalmente si distinguono per due argomenti: religione e politica

\subsection{Religione}
Hegel fa rientrare la religione nell spirito assoluto come forma di conoscenza. Il contenuto della
religione è lo stesso della filosofia

\subsubsection{Sinistra}
Mettono in rilievo che la religione è superata dalla filosofia. Bisogna andare oltre la religione che
è vista come una forma di preparazione alla verità.

\subsubsection{Destra}
Mettono in rilievo la comunanza tra religione e filosofia. La filosofia può e deve avvalorare la 
religione cristiana.

\subsection{Politica}
Hegel ritiente che la storia tenda ad un fine.

\subsubsection{Sinistra}
Non così fedeli alla dialettica hegeliana. Lo stato moderno è una tappa della storia, poi continuerà.
Il mondo non è razionale, bisogna farlo diventare tale. Prevalgono idee democratiche e liberali.

\subsubsection{Destra}
Ciò che è reale è razionale, l'ordine è necessario. La filosofia deve dire la realtà, non criticarla.
Non si deve dire ai governi come funzionare. Prevalgono idee reazionarie sotto la spinta del
congresso di Vienna.

%!TEX ROOT=filosofia.tex

\section{Feuerbach}
Ludwig Feuerbach è il fondatore del \textbf{materialismo filosofico ottocentesco}, nonché anche
esponente della sinistra Hegeliana. GLi scritti fondamentali sono \textit{'Critica della filosofia
Hegeliana'}, \textit{'L'essenza del cristianesimo'} e  \textit{'L'essenza della religione'}.

\subsection{Il rovesciamento dei rapporti di predicazione}
Nel criticare Hegel, Feuerbach critica il rapporto tra concreto e astratto. La natura, dice 
Feuerbach, è materia, natura, non spirito assoluto. Un pensiero simile lo rivolge alla 
\textbf{religione}. La religione parte da un'astrazione (Dio) da cui fa nascere la natura e tutte le 
cose. \textbf{Dio è solo una proiezione degli uomini}. Quindi si rovescia ciò che è scritto nella 
Bibbia. A partire dalla propria visione della vita, gli uomini creano una divinità. Dio ha le
capacità umane elevate alla perfezione.\\
Se si vuole conoscere un popolo si deve conoscere la sua religione perché in essa si esprime la
cultura e il pensiero del popolo. La \textbf{religione è} quindi \textbf{autocoscienza}, indiretta
e capovolta ovvero non si è consapevoli di non conoscere il vero (si crede di conoscere Dio come
vera entità ma non è così!).\\
Se si chiede ad un fedele cosa crede delle altre religioni, dirà che sono invenzioni umane. Feuerbach
fa questo per tutte le religioni.\\
Essere atei non significa negare ogni valore alla religione. Essa infatti è la prima forma di
autocoscienza che è indispensabile.\\
La religione e la filosofia conoscono la stessa cosa per Hegel l'assoluto, per Feuerbach l'uomo.
\begin{description}
  \item[Alienzione religiosa] essere qualcosa che non si è, non riuscire a realizzarsi come uomini,
    l'uomo proietta in Dio sè stesso all'infinito quindi l'uomo punta ad essere Dio e disprezza la
    sua finitezza. \textbf{La religione è pericolosa.}
  \item[Rovesciamento dei rapporti di predicazione] `Rimettere la filosofia con i piedi per terra.'
    Quello che nella religione è il predicato, deve diventare soggetto. (Nella religione `Dio è 
    amore', nella filosofia `L'amore è qualcosa di divino')
\end{description}

%!TEX ROOT=filosofia.tex

\section{Marx}
Karl Marx � il fondatore del comunismo in senso filosofico nonch� un grande conoscitore dell'economia
capitalista. Nel 1844 compone i \textit{'Manoscritti economico-filosofici'}. Nel 1848 pubblica
\textit{'Il manifesto del partito comunista'} in collaborazione con Hengels. Nel 1866 pubblica il
suo scritto principale: \textit{'Il capitale'} (il primo volume).

\subsection{Termini chiave}
\begin{description}
  \item[Ideologia] concezione rovesciata della realt�, presentata come necessaria e materiale. Il
    capitalismo � un'ideologia in quanto crede di essere l'unico e vero sistema economico. Hegel
    credeva che lo stato oggettivasse il bene comune invece � espressione della classe dominante
    che fa i propri interessi.
  \item[Alienazione economica] il capitalismo � alienante nel campo del lavoro
    \begin{description}
      \item[Rispetto al prodotto] il prodotto non � del lavoratore ma del capitalista, il lavoratore
        vede solo una fase della lavorazione.
      \item[Rispetto all'attivit�] il lavoratore nel capitalismo ripete sempre gli stessi gesti,
        senza creativit�, in modo alienante.
      \item[Rispetto al prossimo] il capitalismo induce all'egoismo, riduce i rapporti sociali
        dell'uomo.
    \end{description}
  \item[Alienazione religiosa] gli uomini creano l'alienazione religiosa a causa di quella economica.
    Nella religione cerca una felicit� che non pu� trovare nel lavoro.
\end{description}

\subsection{Critica a Feuerbach}
Feuerbach riteneva che l'uomo fosse natura. Marx gli rimprovera che l'uomo non � solo natura,
\textbf{� anche lavoro}. Si distingue dagli altri esseri viventi per il lavoro. Il lavoro trasforma
il mondo nella storia. Feuerbach � ancora idealista, resta nel campo delle idee, non fa nulla di
pratico.

\subsection{Materialismo storico}
Tutto � mosso da forze economiche. La storia fa i \textbf{modi di produzione}, ovvero l'organizzzione
del lavoro per i beni essenziali.\\
Ci sono due fattori fondamentali della vita sociale e della storia
\begin{description}
  \item[Struttura] base economica della societ�, fatta da forze produttive (=lavoratori, mezzi di
    produzione) e rapporti di produzione (rapporti di propriet� dei mezzi di produzione). Sono
    rapporti determinati dal sistema economico stesso (esistono le classi sociali e quindi diversi
    interessi economici).
  \item[Sovra-struttura] � la cultura, le idee
\end{description}
La sovra-struttura riflette la struttura (la cultura � legata al lavoro economico). Quanto � stretto
questo rapporto?
\begin{itemize}
  \item La struttura determina la sovra-struttura. Il rapporto � necessario, non c'� liberta per
    l'uomo, l'uomo inevitabilmente in quelle situazioni pensa quelle cose
  \item La struttura condizione la sovra-struttura. La influenza.
\end{itemize}
La storia � sempre stata lotta di classe, la struttura economica genera classi diverse con interessi
diversi. Nel capitalismo la lotta di classe si semplifica: borgesia (dirigenti) e proletariato.
La borghesia � stata una classe rivoluzionaria (la elogia) che ha soppiantato la precedente.
Sviluppandosi il capitalismo si sviluppa il proletariato che si prepara a scalzare la borhesia. Da
qui nasce la \textbf{dialettica della storia}: la borghesia crea la sua antitesi (il proletariato)
e assieme creeranno qualcosa di nuovo (il socialismo).

\subsection{Il capitale}
Nel Capitale, Marx critica il \textbf{feticismo delle merci}. La merce viene presentata come qualcosa
di ovvio, scontato nel mercato capitalistico. In realt� sono prodotti umani. Il valore viene affidato
dall'uomo, non bisogna sottomettersi.
\begin{description}
  \item[Merce] � un qualcosa anche immateriale che deve avere
    \begin{description}
      \item[Valore d'uso] deve servire a qualcosa
      \item[Valore di scambio] deve poter essere scambiato con altre merci (misurato dal denaro)
    \end{description}
\end{description}
L'economista cerca l'origine del valore di scambio di una merce. Deve esserci una cosa comune a
tutte le merci: \textbf{il lavoro}. Nasce cos� la teoria del \textit{Valore-Lavoro}: il valore
dipende dal lavoro necessario a produrre una merce, � il lavoro sociale, non di un singolo, �
lavoro medio in quanto varia da societ� a societ� e con il tempo.\\
Nei sistemi \textbf{pre-capitalisti} l'economia funzionava: Merce, vendita, Denaro, acquisto, 
Merce.\\
Nei sistemi \textbf{capitalisti} l'econimia si basava su: Denaro (capitale), investimento, Merce, 
vendita, Denaro (profitto).\\
Da dove viene fuori il profitto? Il valore deriva dal lavoro, non dallo scambio in quanto � equo,
quindi deve derivare dal lavoro. Un lavoratore produce profitto pari al suo salario (= prezzo del
lavoro, una merce) (= al prezzo minimo della vita). Il salario non � pari al valore che produce.
Un lavoratore lavora $n$ ore per pagarsi il salario (\textbf{lavoro necessario}) e il resto genera
\textbf{plus-lavoro} non retribuito. Quindi genera  \textbf{plus-valore}. Il capitalismo � basato
sullo sfruttamento. Il plus-valore non � ancora profitto. Una parte infatti verr� usata per
investimenti (\textbf{capitale costante}) in quanto c'� concorrenza (i salari son il capitale 
variabile).\\
\textbf{Marx pensa di aver trovato cosa metter� in crisi il capitalismo}. Oltre alla lotta di classe,
si cerca sempre di pi� di abbassare il salario ma dopo un certo limite non si pu� andare altrimenti
il lavoratore muore. Si cerca comunque di investire per evitare la concorrenza. L'effetto � quello
di concentrare il capitale in pochissimi uomini (proletarizzazione della borghesia). Avverr� la
\textbf{caduta tendenziale del saggio di profitto}. Il saggio (la percentuale) del profitto rispetto
al capitale tende a diminuire sempre di pi�.

\subsection{Concezione della rivoluzione e del comunismo}
Marx non era utopista. Non ha dato una chiara descrizione di come sar� il comunismo. \textbf{La
rivoluzione avverr�}, implica l'uso della forza e della violenza ma non � necessario. Il passaggio
pu� essere graduale, specialmente nei paesi pi� sviluppati. Ci sono 2 tipi di comunsimo
\begin{description}
  \item[Rozzo] il proletariato prende il potere e lo esercita come classe egemone. Abolisce la
    propriet� privata. Lo stato gestisce l'economia. Il proletariato usa il potere contro la
    borghesia.
  \item[Autentico] stacca completamente dal passato. La propriet� viene completamente abolita. I beni
    non sono pi� dello stato, vengono autonomamente distribuiti a seconda dei bisogni dell'individuo.
    Con lo stato c'era ancora divisione in classi, senza non c'� rischio. Simil-anarchia. Il
    comunismo autentico � ricco, come se non pi� del capitalismo.
\end{description}


\newpage
\listoftodos[Note]
\end{document}

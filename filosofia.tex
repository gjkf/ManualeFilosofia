\documentclass[usenames,a4paper,dvipsnames]{article}
\usepackage[italian]{babel}
\usepackage[utf8]{inputenc}
\usepackage{bm}
\usepackage{amsmath, amsfonts, amssymb, mathtools, calrsfs}
\usepackage[pdftex,dvipsnames]{xcolor}
\usepackage[colorinlistoftodos,prependcaption,textsize=tiny]{todonotes}
\usepackage{xtab, booktabs, array}
\usepackage[labelformat=empty]{caption}
\usepackage{tikz}
\usetikzlibrary{arrows,decorations,calc,intersections,shapes.geometric}
\usepackage{parskip} % To avoid indentation
\usepackage{textcomp} % To have \textcelsius and other symbols
\usepackage{multirow} % For nicer tables with split rows/columns
\usepackage{multicol}
\usepackage{cancel} % Cacelled equations and simplifications
\usepackage{hyperref}
\hypersetup{
    colorlinks,
    citecolor=black,
    filecolor=black,
    urlcolor=black,
    linkcolor=blue
}

\usepackage[showframe=false, top=2cm, bottom=2.5cm, left=2.5cm, right=2.5cm]{geometry}

\usepackage[all]{hypcap}
\usepackage{xargs}
% Some useful TODO commands
\newcommandx{\unsure}[2][1=]{\todo[linecolor=red,backgroundcolor=red!25,bordercolor=red,#1]{#2}}
\newcommandx{\change}[2][1=]{\todo[linecolor=blue,backgroundcolor=blue!25,bordercolor=blue,#1]{#2}}
\newcommandx{\info}[2][1=]{\todo[linecolor=OliveGreen,backgroundcolor=OliveGreen!25, bordercolor=OliveGreen,#1]{#2}}
\newcommandx{\improvement}[2][1=]{\todo[linecolor=Plum,backgroundcolor=Plum!25, bordercolor=Plum,#1]{#2}}
\newcommandx{\thiswillnotshow}[2][1=]{\todo[disable,#1]{#2}}
\setlength{\marginparwidth}{2cm}

% For an older but clearer root. Still \oldsqrt is valid
\usepackage{letltxmacro}
\makeatletter
\let\oldr@@t\r@@t
\def\r@@t#1#2{%
  \setbox0=\hbox{$\oldr@@t#1{#2\,}$}\dimen0=\ht0
  \advance\dimen0-0.2\ht0
  \setbox2=\hbox{\vrule height\ht0 depth -\dimen0}%
  {\box0\lower0.4pt\box2}}
\LetLtxMacro{\oldsqrt}{\sqrt}
\renewcommand*{\sqrt}[2][\ ]{\oldsqrt[#1]{#2} }
\makeatother

% Matrix spacing
\makeatletter
\renewcommand*\env@matrix[1][\arraystretch]{%
  \edef\arraystretch{#1}%
  \hskip -\arraycolsep
  \let\@ifnextchar\new@ifnextchar
  \array{*\c@MaxMatrixCols c}}
\makeatother

\newcommand\twodigits[1]{%
  \ifnum#1<10 0\number#1 \else #1\fi
}
\usepackage[yyyymmdd]{datetime}
\renewcommand{\dateseparator}{-}
\usepackage{fancyhdr}
\pagestyle{fancy}
\fancyhead{} % clear all header fields
\renewcommand{\headrulewidth}{0pt} % no line in header area
\fancyfoot{} % clear all footer fields
\fancyfoot[C,CO]{\thepage}% page number in "outer" position of footer line
\fancyfoot[R,RO]{Copyright \copyright 2017--\the\year$\,$Cossu Davide
} % other info in "inner" position of footer line
\fancyfoot[L,LO]{Version 0.1.2 \today
}
\DeclarePairedDelimiter\norm{\lVert}{\rVert} % ||v||  
\DeclarePairedDelimiter\abs{\lvert}{\rvert} % |v|
\newcommand\markangle[7][red]{% [color] origin A B radius radiusmark mark
  % fill red circle
  \begin{scope}
    \path[clip] (#2) -- (#3) -- (#4);
    \fill[color=#1,fill opacity=0.5,draw=#1,name path=circle]
    (#2) circle (#5);
  \end{scope}
  % middle calculation
  \path[name path=line one] (#2) -- (#3);
  \path[name path=line two] (#2) -- (#4);
  \path[%
    name intersections={of=line one and circle, by={inter one}},
    name intersections={of=line two and circle, by={inter two}}
  ] (inter one) -- (inter two) coordinate[pos=.5] (middle);
  % put mark
  \node at ($(#2)!#6!(middle)$) {#7};
}
\def\mathcolor#1#{\@mathcolor{#1}}
\def\@mathcolor#1#2#3{%
      \protect\leavevmode
      \begingroup
      \color#1{#2}#3%
      \endgroup
}
% For better visual in tables
\renewcommand*{\arraystretch}{2}
% To center with m{}
\newcolumntype{M}[1]{>{\centering\arraybackslash}m{#1}}
\newcommand{\divisor}{\rule{8.7cm}{0.4pt}}

\begin{document}
{
\hypersetup{linkcolor=black}
\tableofcontents
}

%!TEX ROOT=filosofia.tex

\section{Schopenhauer}

Arthur Schopenhauer è un \textbf{romantico critico di Hegel}. Già questo mette in luce una
generale caratterisitca di questo filosofo.\\
La sua opera principale è \textit{Il mondo come volontà e rappresentazione} del 1818. Quest'operà
però porterà successo all'autore solo alla fine degli anni '50 circa.

\subsection{``Il mondo come volontà e rappresentazione''}
Già nel titolo vengono racchiusi i due termini fondamentali per Schopenhauer: \textbf{volontà} e 
\textbf{rappresentazione}. Già la prima frase dell'opera \textit{'Il mondo è una mia 
rappresentazione'} mette in evidenza il distacco dalla filosofia passata. Se non ci si rende conto
di questa verità, non si può fare filosofia. \textbf{Anche la scienza è una rappresentazione.}.

\begin{description}
  \item[Rappresentazione] conoscenza superficiale delle cose, non l'essenza. Per Kant il 
    \textit{fenomeno}. È da fare la distinzione tra Kant e Schopenhauer: Kant credeva che il fenomeno
    fosse una superficie ma comunque reale, per Schopenhauer invece è un'\textit{illusione}, è una
    maschera
\end{description}

Questo limite posto alla scienza è tipicamente romantico, la scienza infatti non può tutto.\\
La rappresentazione implica
\begin{itemize}
  \item Soggetto che osserva
  \item Oggetto che è osservato
\end{itemize}
\textbf{La filosofia ha l'obiettivo di superare la rappresentazione}, di fare metafisica. È opposto
all'atteggiamento Kantiano della filosofia. Come creare però questa metafisica? Si deve partire dal
corpo. Ognuno di noi ha due modi di conoscere il proprio corpo
\begin{itemize}
  \item Rappresentazione come oggetto fra altri oggetti
  \item Intuizione come il proprio corpo, non quello altrui, della volontà di vivere e delle 
    necessità primarie.
\end{itemize}

\begin{description}
  \item[Volontà] è l'essenza del corpo, è la forza ordinatrice. Tutta la natura ha voglia di vivere,
    ogni cosa. Le forze della natura sono manifestazione di questa voglia di vivere. La volontà
    è unica, eterna, infinita e incausata.\\
    La volontà è anche \textit{mancanza}. Se si desidera qualche cosa non lo si ha, è sofferenza.
\end{description}

La felicità, quindi deriva dall'appagamento del desiderio. La vita è come un pendolo che oscilla tra 
dolore causato dalla volontà e la felicità è solo momentanea, causata dall'appagamento di questa
volontà.\\
Il dato reale dell'esistenza è quindi il dolore. Questo rende la filosofia di Schopenhauer 
pessimistica. Proprio per questo punto è stato considerato come un precursore della \textit{'Scuola
del sospetto'}. Con quest'idea della volontà come causa del dolore, Schopenhauer critica l'idea di
Dio della tradizione: se esistesse Dio, sarebbe un essere crudele in quanto l'uomo diventa
consapevole della sofferenza. 
Quindi \textbf{la religione è un'illusione} per nascondere la realtà.

\subsection{La storia}
Schopenhauer \textbf{critica Hegel} per il suo ottimismo: la visione della storia che vuole essere
razionale, è una maschera. In realtà non è razionale, la vita degli uomini è sempre \textit{volontà
di vivere}. I cambiamenti riguardano solo il fenomeno che Schopenhauer vuole superare. Nella natura
umana non è presente benevolenza, ognuno cerca il proprio vantaggio a discapito degli altri (simile
allo stato di natura di Hobbes). \textbf{Lo stato ha il compito di mantenere l'ordine pubblico e 
garantire la proprietà privata.}

\subsection{L'amore}
L'amore è la \textbf{metafisica dell'anima}. L'idea che sia un sentimento che nobilita l'animo
è una maschera. Non c'è altro che l'istinto sessuale, riproduttivo. \textbf{L'uomo che crede di amare
è in realtà schiavo della volontà}.

\subsection{Vie di liberazione dal dolore}
Ci sono delle modalità per liberarsi dal dolore. Il suicidio non è una di queste in quanto sarebbe
arrendersi alla volontà e volere di non volere. Le vie di liberazione dal dolore sono 3:
\begin{description}
  \item[Arte] è sapere e conoscenza superiore alla scienza, quasi filosofia. L'arte conosce le idee,
    le essenze (una scultura rappresenta un valore generale, non quel particolare soggetto). L'arte
    è contemplazione disinteressata. Il dolore termina, ma è momentanea sospensione.
  \item[Morale] nasce da un sentimento, quello della compasssione, della consapevolezza che la 
    sofferenza è comune. Superiamo l'egoismo ed agiamo in modo disinteressato. Nella morale ci
    sono due aspetti:
    \begin{description}
      \item[Giustizia] non fare del male agli altri (virtù negativa)
      \item[Amore] non come \textit{eros} ma come \textit{agape}, fare il bene degli altri (virtù
        positiva)
    \end{description}
  \item[Ascesi] \textit{noluntas}, negazione radicale della volontà. Negare il desiderio sessuale,
    tutti i bisogni, essere poveri per scelta. Una volta raggiunta l'ascesi, non si sa cosa accade in
    quanto è ineffabile, il linguaggio non può descriverlo. Si raggiunge il nulla dei fenomeni,
    una serenità incomprensibile.
\end{description}

%!TEX ROOT=filosofia.tex

\section{Kierkegaard}
Soren Aabye Kierkegaard � un filosofo \textbf{critico di Hegel}. Le sue opere principali sono
\textit{Aut-aut} e \textit{Timore e tremore}. Scriveva per difendere il cristianesimo dagli attacchi,
era critico dei luterani danesi.

\subsection{La categoria del singolo}
In Kierkegaard � fondamentale la categoria del singolo. \textbf{Quello che conta ed � reale � il 
singolo individuo, il popolo, la nazione sono tutte astrazioni.} Il valore della vita dipende 
dall'originalit� del singolo individuo. Rifiuta perci� l'idealismo e il sistemismo: racchiudere in u
unico sistema tutta la realt� � impossibile e insensato.

\subsection{La possibilit�}
Centrale in Kierkegaard � il tema della scelta. La scelta � un \textbf{salto nel vuoto}, la scelta
ci mette di fronte al nulla. Le possibilit� non scelte resteranno nel nulla. Ci sono 3 possibilit�
di fondo, o stadi dell'esistenza
\begin{description}
  \item[Esistenza estetica] Don Giovanni � preso a riferimento. La vita � dedicata al piacere e al
    godimento. Si vive nell'attimo, si vuole evitare la ripetizione. Il godimento � fisico (sessuale)
    e psicologico (della conquista del potere). � destinata alla disperazione in quanto non ha una
    continuit� e un'identit�.
  \item[Esistenza Etica] Giudice Guglielmo � il personaggio. � una vita guidata da valori morali ed
    etici. � marito (continuit�), padre, ha un lavoro onesto. Ha una storia e una personalit�. 
    Giunger� alla tristezza in quando adeguandosi ai valori morali, si uniformer� alla comunit�,
    rifiutando la singolarit�. Si pentir� dei suoi errori.
  \item[Esistenza Religiosa] Abramo � il riferimento. Deve scegliere se sacrificare Isacco, l'ordine
    di Dio � contro la morale, � una scelta irrazionale. La fede quindi � abbandonarsi a Dio senza
    sicurezze e garanzie. � una scelta individuale. Agamennone deve sacrificare Ifigenia. La 
    situazione � diversa perch� ne parla con altri e la scelta � comprensibile (sacrificare la figlia
    per un bene maggiore).
\end{description}
Questi tre stadi non sono compatibili fra di loro. Sono mutualmente esclusivi.

\subsection{L'angoscia}
L'angoscai � la percezione del nulla prima di una scelta. Non � paura. Quando scegliamo siamo di 
fronte al nulla e non ci sono garanzie che la scelta sia giusta. Questa libert� pu� portare al 
peccato.

%!TEX ROOT=filosofia.tex

\section{Correnti post-Hegeliane}
Gli allievi di Hegel si dividono in due correnti: la \textbf{Sinistra} e la \textbf{Destra} 
hegeliana. Principalmente si distinguono per due argomenti: religione e politica

\subsection{Religione}
Hegel fa rientrare la religione nell spirito assoluto come forma di conoscenza. Il contenuto della
religione è lo stesso della filosofia

\subsubsection{Sinistra}
Mettono in rilievo che la religione è superata dalla filosofia. Bisogna andare oltre la religione che
è vista come una forma di preparazione alla verità.

\subsubsection{Destra}
Mettono in rilievo la comunanza tra religione e filosofia. La filosofia può e deve avvalorare la 
religione cristiana.

\subsection{Politica}
Hegel ritiente che la storia tenda ad un fine.

\subsubsection{Sinistra}
Non così fedeli alla dialettica hegeliana. Lo stato moderno è una tappa della storia, poi continuerà.
Il mondo non è razionale, bisogna farlo diventare tale. Prevalgono idee democratiche e liberali.

\subsubsection{Destra}
Ciò che è reale è razionale, l'ordine è necessario. La filosofia deve dire la realtà, non criticarla.
Non si deve dire ai governi come funzionare. Prevalgono idee reazionarie sotto la spinta del
congresso di Vienna.

%!TEX ROOT=filosofia.tex

\section{Feuerbach}
Ludwig Feuerbach è il fondatore del \textbf{materialismo filosofico ottocentesco}, nonché anche
esponente della sinistra Hegeliana. GLi scritti fondamentali sono \textit{'Critica della filosofia
Hegeliana'}, \textit{'L'essenza del cristianesimo'} e  \textit{'L'essenza della religione'}.

\subsection{Il rovesciamento dei rapporti di predicazione}
Nel criticare Hegel, Feuerbach critica il rapporto tra concreto e astratto. La natura, dice 
Feuerbach, è materia, natura, non spirito assoluto. Un pensiero simile lo rivolge alla 
\textbf{religione}. La religione parte da un'astrazione (Dio) da cui fa nascere la natura e tutte le 
cose. \textbf{Dio è solo una proiezione degli uomini}. Quindi si rovescia ciò che è scritto nella 
Bibbia. A partire dalla propria visione della vita, gli uomini creano una divinità. Dio ha le
capacità umane elevate alla perfezione.\\
Se si vuole conoscere un popolo si deve conoscere la sua religione perché in essa si esprime la
cultura e il pensiero del popolo. La \textbf{religione è} quindi \textbf{autocoscienza}, indiretta
e capovolta ovvero non si è consapevoli di non conoscere il vero (si crede di conoscere Dio come
vera entità ma non è così!).\\
Se si chiede ad un fedele cosa crede delle altre religioni, dirà che sono invenzioni umane. Feuerbach
fa questo per tutte le religioni.\\
Essere atei non significa negare ogni valore alla religione. Essa infatti è la prima forma di
autocoscienza che è indispensabile.\\
La religione e la filosofia conoscono la stessa cosa per Hegel l'assoluto, per Feuerbach l'uomo.
\begin{description}
  \item[Alienzione religiosa] essere qualcosa che non si è, non riuscire a realizzarsi come uomini,
    l'uomo proietta in Dio sè stesso all'infinito quindi l'uomo punta ad essere Dio e disprezza la
    sua finitezza. \textbf{La religione è pericolosa.}
  \item[Rovesciamento dei rapporti di predicazione] `Rimettere la filosofia con i piedi per terra.'
    Quello che nella religione è il predicato, deve diventare soggetto. (Nella religione `Dio è 
    amore', nella filosofia `L'amore è qualcosa di divino')
\end{description}

%!TEX ROOT=filosofia.tex

\section{Marx}
Karl Marx � il fondatore del comunismo in senso filosofico nonch� un grande conoscitore dell'economia
capitalista. Nel 1844 compone i \textit{'Manoscritti economico-filosofici'}. Nel 1848 pubblica
\textit{'Il manifesto del partito comunista'} in collaborazione con Hengels. Nel 1866 pubblica il
suo scritto principale: \textit{'Il capitale'} (il primo volume).

\subsection{Termini chiave}
\begin{description}
  \item[Ideologia] concezione rovesciata della realt�, presentata come necessaria e materiale. Il
    capitalismo � un'ideologia in quanto crede di essere l'unico e vero sistema economico. Hegel
    credeva che lo stato oggettivasse il bene comune invece � espressione della classe dominante
    che fa i propri interessi.
  \item[Alienazione economica] il capitalismo � alienante nel campo del lavoro
    \begin{description}
      \item[Rispetto al prodotto] il prodotto non � del lavoratore ma del capitalista, il lavoratore
        vede solo una fase della lavorazione.
      \item[Rispetto all'attivit�] il lavoratore nel capitalismo ripete sempre gli stessi gesti,
        senza creativit�, in modo alienante.
      \item[Rispetto al prossimo] il capitalismo induce all'egoismo, riduce i rapporti sociali
        dell'uomo.
    \end{description}
  \item[Alienazione religiosa] gli uomini creano l'alienazione religiosa a causa di quella economica.
    Nella religione cerca una felicit� che non pu� trovare nel lavoro.
\end{description}

\subsection{Critica a Feuerbach}
Feuerbach riteneva che l'uomo fosse natura. Marx gli rimprovera che l'uomo non � solo natura,
\textbf{� anche lavoro}. Si distingue dagli altri esseri viventi per il lavoro. Il lavoro trasforma
il mondo nella storia. Feuerbach � ancora idealista, resta nel campo delle idee, non fa nulla di
pratico.

\subsection{Materialismo storico}
Tutto � mosso da forze economiche. La storia fa i \textbf{modi di produzione}, ovvero l'organizzzione
del lavoro per i beni essenziali.\\
Ci sono due fattori fondamentali della vita sociale e della storia
\begin{description}
  \item[Struttura] base economica della societ�, fatta da forze produttive (=lavoratori, mezzi di
    produzione) e rapporti di produzione (rapporti di propriet� dei mezzi di produzione). Sono
    rapporti determinati dal sistema economico stesso (esistono le classi sociali e quindi diversi
    interessi economici).
  \item[Sovra-struttura] � la cultura, le idee
\end{description}
La sovra-struttura riflette la struttura (la cultura � legata al lavoro economico). Quanto � stretto
questo rapporto?
\begin{itemize}
  \item La struttura determina la sovra-struttura. Il rapporto � necessario, non c'� liberta per
    l'uomo, l'uomo inevitabilmente in quelle situazioni pensa quelle cose
  \item La struttura condizione la sovra-struttura. La influenza.
\end{itemize}
La storia � sempre stata lotta di classe, la struttura economica genera classi diverse con interessi
diversi. Nel capitalismo la lotta di classe si semplifica: borgesia (dirigenti) e proletariato.
La borghesia � stata una classe rivoluzionaria (la elogia) che ha soppiantato la precedente.
Sviluppandosi il capitalismo si sviluppa il proletariato che si prepara a scalzare la borhesia. Da
qui nasce la \textbf{dialettica della storia}: la borghesia crea la sua antitesi (il proletariato)
e assieme creeranno qualcosa di nuovo (il socialismo).

\subsection{Il capitale}
Nel Capitale, Marx critica il \textbf{feticismo delle merci}. La merce viene presentata come qualcosa
di ovvio, scontato nel mercato capitalistico. In realt� sono prodotti umani. Il valore viene affidato
dall'uomo, non bisogna sottomettersi.
\begin{description}
  \item[Merce] � un qualcosa anche immateriale che deve avere
    \begin{description}
      \item[Valore d'uso] deve servire a qualcosa
      \item[Valore di scambio] deve poter essere scambiato con altre merci (misurato dal denaro)
    \end{description}
\end{description}
L'economista cerca l'origine del valore di scambio di una merce. Deve esserci una cosa comune a
tutte le merci: \textbf{il lavoro}. Nasce cos� la teoria del \textit{Valore-Lavoro}: il valore
dipende dal lavoro necessario a produrre una merce, � il lavoro sociale, non di un singolo, �
lavoro medio in quanto varia da societ� a societ� e con il tempo.\\
Nei sistemi \textbf{pre-capitalisti} l'economia funzionava: Merce, vendita, Denaro, acquisto, 
Merce.\\
Nei sistemi \textbf{capitalisti} l'econimia si basava su: Denaro (capitale), investimento, Merce, 
vendita, Denaro (profitto).\\
Da dove viene fuori il profitto? Il valore deriva dal lavoro, non dallo scambio in quanto � equo,
quindi deve derivare dal lavoro. Un lavoratore produce profitto pari al suo salario (= prezzo del
lavoro, una merce) (= al prezzo minimo della vita). Il salario non � pari al valore che produce.
Un lavoratore lavora $n$ ore per pagarsi il salario (\textbf{lavoro necessario}) e il resto genera
\textbf{plus-lavoro} non retribuito. Quindi genera  \textbf{plus-valore}. Il capitalismo � basato
sullo sfruttamento. Il plus-valore non � ancora profitto. Una parte infatti verr� usata per
investimenti (\textbf{capitale costante}) in quanto c'� concorrenza (i salari son il capitale 
variabile).\\
\textbf{Marx pensa di aver trovato cosa metter� in crisi il capitalismo}. Oltre alla lotta di classe,
si cerca sempre di pi� di abbassare il salario ma dopo un certo limite non si pu� andare altrimenti
il lavoratore muore. Si cerca comunque di investire per evitare la concorrenza. L'effetto � quello
di concentrare il capitale in pochissimi uomini (proletarizzazione della borghesia). Avverr� la
\textbf{caduta tendenziale del saggio di profitto}. Il saggio (la percentuale) del profitto rispetto
al capitale tende a diminuire sempre di pi�.

\subsection{Concezione della rivoluzione e del comunismo}
Marx non era utopista. Non ha dato una chiara descrizione di come sar� il comunismo. \textbf{La
rivoluzione avverr�}, implica l'uso della forza e della violenza ma non � necessario. Il passaggio
pu� essere graduale, specialmente nei paesi pi� sviluppati. Ci sono 2 tipi di comunsimo
\begin{description}
  \item[Rozzo] il proletariato prende il potere e lo esercita come classe egemone. Abolisce la
    propriet� privata. Lo stato gestisce l'economia. Il proletariato usa il potere contro la
    borghesia.
  \item[Autentico] stacca completamente dal passato. La propriet� viene completamente abolita. I beni
    non sono pi� dello stato, vengono autonomamente distribuiti a seconda dei bisogni dell'individuo.
    Con lo stato c'era ancora divisione in classi, senza non c'� rischio. Simil-anarchia. Il
    comunismo autentico � ricco, come se non pi� del capitalismo.
\end{description}

%!TEX ROOT=filosofia.tex

\section{Positivismo}
Il positivismo si sviluppa tra la seconda metà dell'800 e gli inizi del '900.
\begin{description}
  \item[Positivo] ciò che è conosciuto in modo diretto, con esperienza. Anche come utile, applicabile
    praticamente.
\end{description}
La scienza è l'unico modo per conoscere la natura ed è un sapere utile per il progresso storico e
tecnico dell'uomo. Viene rifiutata la metafisica romantica, solo i fenomeni sono utili ed esistono.
Viene rifiutata anche la religione, vista come sapere astratto e inutile. Bisogna estendere il
metodo della scienza in tutti i campi, nascono così Psicologia e Sociologia (Comte).

\subsection{Rapporto con l'Illuminismo}
Sia il positivismo che l'Illuminismo hanno in comune
\begin{itemize}
  \item La fiducia nella ragione e nel sapere, visti come mezzo di progresso
  \item Esaltazione della scienza a scapito della metafisica
  \item La visione laica e immanentistica della vita
\end{itemize}
Invece differiscono su altri punti come
\begin{itemize}
  \item Il momento storico è molto diverso e quindi il positivismo manca di una carica polemica
    che era presente nell'Illuminismo (la borghesia ormai si è affermata). Il positivismo è una
    forza riformista consapevolmente anti-rivoluzionaria
  \item La filosofia è vista in modo diverso: gli illuministi la consideravano come una critica
    della scienza, una visione gnoseologica, i positivisiti invece affidano alla filosofia il compito
    di ordinare le scienze e unificarle
  \item La scienza è vista nel positivismo come un sapere assoluto, senza limiti. Nell'Illuminismo
    invece con Hume o Kant erano stati posti dei paletti che la scienza non poteva valicare
\end{itemize}

\subsection{Rapporto con il Romanticismo}
Nonostante ci siano molte differenze tra le due correnti, si possono fare alcune analogie. 
Innanzitutto le differenze principali sono
\begin{itemize}
  \item Il Romanticismo parla in termini di \textit{spirito}, \textit{assoluto}, il Positivismo 
    invece di scienza, Umanità e progresso
  \item Il Romanticismo è espressione di una società pre-industriale, il positivismo è di una
    capitalistica
\end{itemize}
Come somiglianze si può considerare il Positivismo come \textit{romanticismo della scienza}, 
l'esaltazione del sapere positivo.

%!TEX ROOT=filosofia.tex

\section{Comte}
Auguste Comte è il fondatore del positivismo in Francia, nonché fondatore/ideatore della sociologia.
La sua opera principale è \textit{'Corso di filosofia positiva'}. 

\subsection{Legge dei tre stadi}
Comte pensa di aver fatto una scoperta: la \textbf{Legge dei tre stadi} che è una filosofia della
storia e della conoscenza. Sono tre stadi che valgono per l'umanita e il singolo. I seguenti
sono i tre stadi
\begin{description}
  \item[Stadio Teologico] l'uomo è guidato dalla \textbf{fantasia e immaginazione}. Le cause dei
    fenomeni sono soprannaturali. L'epoca storica di riferimento è il \textbf{Medioevo} dominato
    da re e principi, dalla Chiesa e dalla religione. La forma di governo è la \textbf{Monarchia}.
    È un'epoca \textbf{organica}, caratterizzata da ordine e stabilità.
  \item[Stadio Metafisico] è guidato dalla \textbf{ragione} che cerca le cause dei fenomeni, oltre
    i fenomeni (essenza, forma, \ldots). Corrisponde all'\textbf{Età moderna}. È un'epoca 
    \textbf{critica}, caratterizzata da rivoluzioni, cambiamenti, disordini.
  \item[Stadio positivo] ancora non è realizzato, Comte se lo aspetta in un futuro prossimo (visione
    finalistica della storia). La \textbf{scienza} guida il popolo. Perché si arrivi a questo stadio
    (che è \textbf{organico}) è necessario che esista la \textbf{sociologia}. Al potere saranno
    i tecnici che governeranno per il bene comune applicando leggi scientifiche. Il potere
    culturale lo hanno gli scienziati, non è democrazia (le idee fondanti della democrazia sono
    metafisiche, astratte. L'uomo non deve pensare ai diritti, ma ai doveri della società. Gli uomini
    non sono liberi o uguali, la scienza è una sola).
\end{description}

\subsection{Le scienze}
Ogni scienza identifica delle \textbf{leggi} a partire dall'osservazione dei fenomeni. Queste leggi
sono ciò che rende utile la scienza perché consentono di prevedere i fenomeni futuri.\\
Comte voleva arrivare ad una \textbf{classificazione delle scienze}. Uno dei caratteri fondamentali
è la specializzazione delle scienze: Comte non è contrario a questa pratica ma teme si possa perdere
la visione d'insieme. Proprio questo è il compito della filosofia. \textbf{La filosofia deve capire
un metodo scientifico mantenendo la visione generale}. Per Comte si raggiunge la scientificità di una
pratica tanto prima tanto è più generale. E più è generale più è facile. Secondo Comte l'ordine è
Matematica, Fisica, Astronomia, Chimica, Biologia, \ldots. Mancano però due cose: \textbf{Psicologia
e Sociologia}.

\subsubsection{Psicologia}
La psicologia non potrà mai diventare una scienza perché dovrebbe essere basata 
sull'auto-osservazione. Viene quindi meno l'oggettività necessaria per una scienza. Ci sono già delle
scienze che studiano l'uomo: la Biologia e la Sociologia

\subsubsection{Sociologia}
Per capire l'uomo bisogna conoscere la società in quanto l'individuo ne fa parte. Si può dividere in
due
\begin{description}
  \item[Statica] cosa permette la stabilità della società (proprietà privata, famiglia, potere)
  \item[Dinamica] cosa permette il progresso della società (la legge dei tre stadi)
\end{description}

%!TEX ROOT=filosofia.tex

\section{Mill}
John Stuart Mill è un positivista inglese il cui principale scritto è \textit{``Sistema di logica 
deduttiva ed induttiva''}. 

\subsection{Sistema di logica deduttiva ed induttiva}
Mill è un \textbf{empirista radicale}, si rifà a Locke e a Hume: ogni cosa, anche la più astratta,
deriva dall'esperienza.\\
Mill in particolar modo si occupa dell'induzione: in un modo o nell'altro
si deve partire dall'esperienza, anche le premesse di un sillogismo lo fanno. Cosa ci permette di
passare dal particolare all'universale? \textbf{L'uniformità della natura}. Ovvero che a cause simili
corrispondono effetti simili. Questo non è dato a priori, si ricava anch'esso dall'esperienza. Si
può qui entrare in un un circolo vizioso: dall'induzione si trova il principio di causa che trova
l'induzione e così via. Fin'ora però non è mai stato smentito che una causa c'è sempre. Però
\textbf{non è possibile escludere che ci siano fatti indeterminati}. La scienza quindi è fallibile. 

\subsection{Etica}
Mill è un \textbf{utilitarista} radicale (benesserismo, consequenzialismo, aggregazionismo). Mill
però si allontana un po' dall'utilitarismo classico in quanto ritiene che ci siano diversi tipi di
piaceri, di diverse qualità. \textbf{L'altruismo è il piacere più alto di tutti}. Questo dimostra
che l'utilitarismo può andare senza problemi assieme al Cristianesimo.\\
L'uomo deve essere lasciato libero di agire ed essere felice fino a che le conseguenze delle sue 
azioni non ricadono su altri individui. Solo in questo caso lo Stato può limitare la libertà. 
\textbf{Sul proprio pensiero e sul proprio corpo, l'individuo è sovrano}. In argomenti di bio-etica
si mantiene la stessa linea di pensiero: se non danneggia altri, si è liberi. Nell'aborto l'embrione
non è razionale, l'unico essere razionale è la madre, quindi l'aborto è consentito.

\subsection{Politica}
Mill è un \textbf{liberale} secondo cui si deve rispettare la legge per evitare di danneggiare una
minoranza. È un male necessario, però le leggi devono lasciare la massima libertà.
\begin{itemize}
  \item Un po' limita la libertà individuale
  \item È necessario per evitare di danneggiare altri
\end{itemize}
Lo Stato dev'essere coercitivo solo per evitare che gli uomini si danneggino fra di loro.\\
Mill crede che \textbf{ci debba essere libertà di religione e di idee} in maniera assoluta. Il 
progresso della storia deriva dalla libertà individuale, le idee dei singoli devono essere espresse,
altrimenti non ci sarà progresso. Anche chi, in minoranza, crede in idee sbagliate deve essere
lasciato libero perché si deve ridiscutere la propria idea e da ciò nasce l'innovazione. Mill
inoltre tratta il tema dell'\textbf{emancipazione femminile} in alcuni suoi saggi, fra cui 
\textit{``L'asservimento delle donne''}. La società non deve intromettersi nei sentimenti di una
coppia. La donna nella società di Mill vive una condizione simile alla schiavitù, con dei limiti 
nelle professioni, nelle libertà di scelta (prima di vendere un bene, doveva chiederlo al marito). 
Per la donna la migliore condizione era la vedovanza. In un aspetto la situazione era peggiore degli
schiavi: se con essi il legame era evidente, con la donna no, sono infatti educate inconsapevolmente
sin da piccole alla loro inferiorità. Viene fatto passare come qualcosa di conveniente alla donna.
Bisogna cambiare la mentalità della società e dell'istruzione. Secondo Mill, se liberiamo le
donne da questa schiavitù, liberiamo anche la loro intelligenza e quindi ci sarà progresso, 
innovazione.

%!TEX ROOT=filosofia.tex

\section{Lamarck}
Lamarck è uno dei primi filosofi \textbf{evoluzionisti}. Formula la così detta \textit{``Teoria della
trasformazione delle specie''}.\\
Gli organismi vivono all'interno di un \textbf{ambiente} e per sopravvivere sviluppano più o meno
alcuni arti. L'uso e il disuso di questi arti porta all'evoluzione. Lamarck è un \textbf{sostenitore
dell'ereditarietà dei caratteri acquisiti.} Infatti le trasformazioni avvengono per l'intera
popolazione, non per il singolo individuo. La visione è puramente meccanicista.

%!TEX ROOT=filosofia.tex

\section{Cuvier}
Cuvier era uno dei migliori paleontologi del tempo. Era un forte critico delle teorie lamarckiane.\\
Cuvier riteneva che se un organo muta, si deve adattare tutto il corpo. È necessario quindi un
dio che ordini e organizzi tutto quanto. Ma come si spiegano i fossili che si ritrovano? Tramite
il \textbf{catastrofismo}, essi infatti non sono altro che specie passate che ora sono state spazzate
via da catastrofi naturali.

%!TEX ROOT=filosofia.tex

\section{Lyell}
Lyell era un amico di penna di Darwin. È un sostenitore dell'\textbf{uniformismo}.\\
Il paesaggio infatti è causa dell'azione costante e prolungata degli agenti atmosferici che lo hanno
plasmato in questo modo. La Terra dunque deve essere per forza molto più vecchia di quello che le 
sacre scritture dicevano.

%!TEX ROOT=filosofia.tex

\section{Darwin}
Darwin è il padre dell'evoluzionismo biologico. Era di famiglia benesatante con un forte impatto
scientifico (erano molti naturalisti, medici). La sua massima opera è \textit{``L'origine delle 
specie''}.\\
Nel 1831 si imbarca in un viaggio finanziato dal governo inglese che aveva l'obiettivo di esplorare
il Sud America e le isole circostanti. Il viaggio duro \textbf{5 anni} nei quali Darwin raccolse 
molti dati ed informazioni su cui poi lavorerà tutta la vita.

\subsection{``L'origine delle Specie''}
Fu influenzato dallo scritto di Malthus \textit{``Saggio sul principio di popolazione''} in cui 
esprimeva l'idea di uno squilibrio tra risorse e popolazione. Questo squilibrio porterà ad una lotta
per la sopravvivenza.\\
L'evoluzione è causata dalle \textbf{piccole variazioni} che sono i cambiamenti naturali che si 
vedono tra genitori e figli. Alle Galapagos Darwin ebbe il modo di criticare Lamarck in quanto nello
stesso ambiente, si potevano vedere specie fondamentalmente diverse. Le piccole variazioni sono sia 
favorevoli che sfavorevoli e sopratutto \textbf{non si ereditano i caratteri acquisiti}. L'ambiente
non ha una funzione di selezione, non è causa dell'evoluzione. C'è invece una \textbf{lotta per la
sopravvivenza} che invece seleziona le specie. La lotta è sia tra specie diverse che all'interno 
della stessa specie.\\ [\baselineskip]
Ricevette grandi elogi ma anche forti critiche in quanto si toccavano questioni delicate e Darwin
non sapeva spiegare le piccole variazioni e come mai si presentassero. Alcune critiche furono mosse
da Kelvin che gli rimproverava che la Terra era troppo giovane perché l'evoluzione potesse essere
credibile (questa teoria era sbagliata). Darwin la prende molto sul serio e modifica leggermente la 
sua idea dicendo che anche l'ambiente può accelerare il processo.

\subsection{Il rapporto uomo-animale}
Non ne parla mai nell'Origine delle Specie, ma in altri scritti più tardi. L'uomo è un essere 
naturale che è sottoposto alle stesse leggi degli animali. L'uomo non differisce dagli animali per
qualità, ma per grado. Infatti anche gli animali hanno una certa intelligenza, solo di grado 
inferiore all'uomo. Lyell, Wallace e altri credono che la \textbf{morale} differisca l'uomo dagli 
animali in quanto non è spiegabile nell'evidenza biologica. Darwin invece crede che sia solo una
strategia di sopravvivenza. Darwin non è finalista, non crede ci sia un Dio buono in un mondo così
sofferente.

%!TEX ROOT=filosofia.tex

\section{Spencer}
Spencer è un filosofo positivista evoluzionista con una concezione che racchiude sia Darwin che 
Lamarck.

\subsection{Rapporti scienza-religione}
Spencer definisce \textbf{l'Inconoscibile}, ovvero l'inacessibilità della realtà ultima e assoluta.
Questa inaccessibilità mette su di un piano comune la religione e la scienza.\\
In ogni religione la verità ultima è esprimibile come ``l'esistenza del mondo è un mistero che va
interpretato'', però ogni religione fallisce nell'interpretarlo in quanto non ha delle dimostrazioni
logiche. Di conseguenza, \textbf{la religione riconosce che il mistero della natura è imperscrutabile
e ciò che ``Inconoscibile''}.\\
Anche la scienza nella sua ricerca si scontra con delle domande che sono impenetrabili (cosa sia il
tempo, lo spazio, \ldots). \textbf{Le idee scientifiche sono quindi rappresentative di realtà
incomprensibili}.\\
La nostra conoscenza è chiusa entro dei limiti del relativo, il progresso consiste nell'includere
verità sempre maggiori che contenevano le precedenti. \textbf{La verità assoluta non può essere
inclusa in un'altra, quindi è destinata ad essere un mistero}. L' \textbf{Assoluto} quindi è la
forza misteriosa che si manifesta in tutti i fenomeni. Poiché non si può trovare una causa di questa
forza, la religione \textbf{richiamerà il mistero che rappresenta}, la scienza \textbf{estende la
conoscenza fino a questo limite}.\\
Il fenomeno quindi è la manifestazione di questo Inconoscibile. Ogni nozione persistente e immutabile
deiva quindi dall'Inconoscibile, ne è un suo modo di esprimersi. Questa corrispondenza è il
\textbf{realismo trasfigurato}. 

\subsection{Teoria dell'evoluzione}
Qual è il compito della filosofia? La filosofia è \textbf{la conoscenza nel suo più alto grado di 
generalità}. È una conoscenza unificata. Quindi pone come base i principi più ampli a cui la scienza
è giunta. Essi sono
\begin{itemize}
  \item L'indistruttibilità della materia
  \item La continuità del movimento
  \item La persistenza della forza
\end{itemize}
A questi si deve aggiungere la \textbf{legge del ritmo}, ovvero il ciclico alternarsi di fasi acute e
di fasi di caduta.\\
Questi principi richiedono una legge che combini continuamente la materia, essa è 
\textbf{l'evoluzione} secondo cui
\begin{description}
  \item[Si passa dall'incoerente al coerente]
  \item[Si passa dall'omogeneo all'eterogeneo] Ogni organismo prima si sviluppa attraverso la 
    differenziazione delle sue parti, poi si diversificano ulteriormente in tessuti e organi
  \item[Si passa dall'indefinito al definito] Dal vago al preciso
\end{description}
In questo la materia passa da uno stato di dispersione ad uno di integrazione, la forza invece si 
dissipa. L'evoluzione è un passaggio \textbf{necessario} in quanto l'omogeneità è instabile. È 
inoltre necessariamente migliorativo. Anche se per la legge del ritmo ci saranno momenti di
caduta, sono sempre premesse per un'ulteriore evoluzione.

\subsection{Biologia e Psicologia}
La Biologià è lo studio dell'evoluzione dei fenomeni organici. \textbf{La vita è una funzione 
dell'adattamento grazie alla quale organi si formano e si differenziano}. Segue Lamarck secondo cui è
la funzione a creare l'organo, ma segue anche la selezione naturale. Il progresso della vita è
quindi un continuo adattamento all'ambiente.\\ [\baselineskip]
La Psicologia è possibile come scienza autonoma. Ce ne sono di due tipi
\begin{description}
  \item[Ogettiva] che studia i fenomeni psichici
  \item[Soggettiva] che si fonda sull'introspezione
\end{description}
Soltanto la soggettiva può contribuire allo sviluppo del pensiero come adattamento graduale. Spencer
inoltre da anche delle \textbf{nozioni a priori} che sono uniche per l'individuo e non comuni alla
specie.

\subsection{Sociologia e politica}
La sociologia di Spencer è molto diversa da quella di Comte. Infatti per Comte era la massima scienza
quando per Spencer \textbf{deve limitarsi a descrivere lo sviluppo della società umana fino al
presente}. Può studiare le condizioni per lo sviluppo, ma non le mete a cui ambisce che sono invece
definite dalla morale.\\ [\baselineskip]
Spencer si incentra sulla \textbf{difesa delle libertà individuali} e questo lo orienta verso un 
certo individualismo. Lo sviluppo della società dev'essere affidato alla forza spontanea che lo muove
verso il progresso, l'intervento dello Stato rallenta e basta. \textbf{Lo sviluppo sociale è graduale
e inevitabile}. Lo stesso sviluppo sociale ha determinato il passaggio da una cooperazione umana
imposta ad una più libera e spontanea. Questo è il passaggio dal \textbf{regime militare} (prevalenza
del potere statale sugli individui) al \textbf{regime industriale} (fondato sull'indipendenza degli
indivudui). È possibile un terzo regime sociale in cui egoismo e altruismo convivono. 

\subsection{Etica evoluzionistica}
L'etica è biologica, ha per oggetto l'adattamento progressivo dell'uomo alle sue condizioni di vita.
\textbf{L'adattamento non è solo un miglioramento ma è un raggiungimento di maggiore intensità e 
ricchezza della vita}. Il bene si identifica con il piacere.\\
L'uomo singolo agisce per dovere: per un sentimento di obligazzione morale generato da esperienze che
hanno prodotto nell'uomo un sentimento per cui questo sembra più utile per il raggiungimento del 
benessere. \textbf{Il senso di dovere è transitorio} e il progresso e la crescita dell'uomo fanno
scomparire questi obblighi trasformandoli in gesti di altruismo. Questo significa che 
\textbf{altruismo ed egoismo possono essere in perfetto accordo}. 


\newpage
\listoftodos[Note]
\end{document}

%!TEX ROOT=filosofia.tex

\section{Correnti post-Hegeliane}
Gli allievi di Hegel si dividono in due correnti: la \textbf{Sinistra} e la \textbf{Destra} 
hegeliana. Principalmente si distinguono per due argomenti: religione e politica

\subsection{Religione}
Hegel fa rientrare la religione nell spirito assoluto come forma di conoscenza. Il contenuto della
religione è lo stesso della filosofia

\subsubsection{Sinistra}
Mettono in rilievo che la religione è superata dalla filosofia. Bisogna andare oltre la religione che
è vista come una forma di preparazione alla verità.

\subsubsection{Destra}
Mettono in rilievo la comunanza tra religione e filosofia. La filosofia può e deve avvalorare la 
religione cristiana.

\subsection{Politica}
Hegel ritiente che la storia tenda ad un fine.

\subsubsection{Sinistra}
Non così fedeli alla dialettica hegeliana. Lo stato moderno è una tappa della storia, poi continuerà.
Il mondo non è razionale, bisogna farlo diventare tale. Prevalgono idee democratiche e liberali.

\subsubsection{Destra}
Ciò che è reale è razionale, l'ordine è necessario. La filosofia deve dire la realtà, non criticarla.
Non si deve dire ai governi come funzionare. Prevalgono idee reazionarie sotto la spinta del
congresso di Vienna.

%!TEX ROOT=filosofia.tex

\section{Nietsche}
Wilhelm Nietsche è uno dei così detti \textbf{maestri del sospetto}. Il suo ``destino'' è sempre 
stato quello di criticare ciò che fin quel momento era considerato \textbf{sacro} (i valori della
tradizione) mostrando che in realtà sono maschere. Nietsche \textbf{non è solo} un 
\textbf{pessimista}, solo a partire dal suo pensiero ci sono speranze.

\subsection{``La nascita della Tragedia''}
Nelle tragedie greche si incontrano due fattori comuni della cultura:
\begin{description}
  \item[Spirito apollineo] dal dio Apollo, simbolo di ordine e razionalità
  \item[Spirito dionisiaco] dal dio Dioniso, simbolo di perdita della ragione ed istinto
\end{description}
Nella tragedia i due spiriti convivono (la vicenda e i dialoghi sono apollinei, i canti e la musica
dionisiaci). Questo è il \textbf{miracolo greco} e segna quindi il vertice della cultura. Lo spirito
della tragedia si è andato perdendo, ora potrebbe tornare fuori con Wagner.\\ [\baselineskip]
Lo spirito \textbf{dionisiaco} si ritrovava anche nella filosofia da \textbf{Eraclito}, da 
\textbf{Socrate} prende il sopravvento lo spirito \textbf{apollineo} e così comincia la decadenza.
Socrate svalorizzava il corpo, valorizzando l'anima. Nietsche vuole valorizzare il corpo, la vita
e quindi disprezza la religione e la scienza che valorizzano l'anima. L'arte può cogliere l'essenza 
della vita.

\subsection{``Considerazioni inattuali''}
Nietsche credeva di essere inattuale, di avere anticipato i tempi. Sostiene 4 considerazioni
\begin{enumerate}
  \item Schopenhauer è il suo educatore (ha smascherato le illusioni)
  \item Wagner riprenderà il miracolo greco
  \item Strauss (sinistra hegeliana)
  \item \textit{``Sull'inutilità e il danno dello studio della storia''}: non è solo importante la
    memoria, ma anche l'oblio, lo studio della storia distoglie dalle passioni e dalle azioni
\end{enumerate}

\subsection{Prospettivismo}
Dal '78 al '82 Nietsche vive il \textbf{periodo illuministico} così denominato perché ammira gli
illuministi e la loro critica della religione. Non è pienamente illuminista in quanto non ha
fiducia nella scienza.\\
Secondo Nietsche la scienza è un punto di vista, una \textbf{prospettiva} che ci fa conoscere ciò che
è oggettivo e misurabile. La musica infatti, una volta misurata cosa è? Nulla, niente emozioni quando
in realtà non è così. \textbf{Non esiste quindi una verità assoluta: non ci sono fatti, solo 
interpretazioni}. 

\subsection{La ``Gaia scienza'' e ``Così parlò Zarathustra''}
\subsubsection{La morte di Dio}
In un aforisma all'interno della ``Gaia Scienza'' intitolato \textit{L'uomo folle}, Nietsche descrive
la \textbf{morte di Dio}. Un uomo folle accende una lanterna nel mezzo del giorno e va alla ricerca
di Dio dove ci sono persone che non credono in Dio. Questi lo prendono in giro. L'uomo allora capisce
che Dio è morto.\\
La morte di Dio non è una dimostrazione dell'ateismo, l'ateismo è ovvio, scontato. ``Dio è morto e
noi l'abbiamo ucciso''. Com'è stato possibile? Come sarà ora la nostra esistenza? La morte di Dio
prevede la perdita dei valori morali, l'umanità non crede più in Dio. \textbf{Gli uomini hanno
creato Dio per sopportare i mali della vita, ora che lo hanno ucciso, saranno all'altezza?} Sapranno
vivere senza Dio? Non dovranno diventare dei a loro volta per fare ciò? \textbf{Devono diventare
qualcosa di nuovo, di diverso dal passato}.\\
Dio in questo racconto rappresenta non solo il Dio della tradizione, ma anche tutti i valori e gli
ideali occidentali. \textbf{La storia} quindi \textbf{non è progresso, ma decadenza} in cui i modelli
diventeranno sempre meno credibili. Avviene quindi l'\textbf{autosoprressione della morale}, che è
la situazione dell'uomo contemporaneo.
\paragraph{``Crepuscolo degli idoli'', ``Come il mondo vero finì per diventare favola''}
In questo aforisma, Nietsche descrive la storia come decadenza e ne indivuda 6 momenti
\begin{description}
  \item[Platone] che conosce la verità
  \item[Cristianesimo] che promette la verità ai virtuosi
  \item[Kant] e la sua idea di una verità assoluta non realizzabile, vista come dovere
  \item[Positivismo] che critica tutto ciò che non è scienza
  \item[Nietsche] che rompe con il passato e l'idea diventa inutile
  \item[Nietsche] che ha eliminato anche il mondo apparente, rimangono solo le prospettive
\end{description}

\subsubsection{Il Superuomo}
La crisi causata dalla morte di Dio ha due possibili effetti
\begin{description}
  \item[L'ultimo uomo] descritto nella prefazione a ``Così parlò Zarathustra'' che si dispera. È 
    l'uomo della società di massa, l'uomo della tradizione
  \item[Il Superuomo] descritto in ``Così parlò Zarathustra'' che possiede 3 metamorfosi
    \begin{description}
      \item[Cammello] è l'uomo della tradizione che acectta la sofferenza
      \item[Leone] è Zarathustra che si contrappone ai valori tradizionali e alla dottrina del 
        ``tu devi''
      \item[Fanciullo] è il vero Superuomo capace di creare i suoi valori
    \end{description}
\end{description}
Se l'ultimo uomo si dispera e disprezza la vita, il Superuomo crea nuovi valori e valorizza la vita.
Ci sono state varie interpretazioni del Superuomo niestchano, una delle più particolari è quella di
\textbf{Vattimo} secondo cui Übermensch sia da tradurre come \textbf{Oltreuomo} perché innova,
va oltre l'uomo passato. Inoltre Nietsche sarebbe da essere interpretato come Marx, come filosofo
di liberazione.

\subsubsection{Eterno ritorno}
Viene presentato nella ``Gaia Scienza'' nell'aforisma ``Il peso più grande'' in cui ad un certo punto
un demone sottoforma di seprente dice al protagonista che \textbf{il mondo è un ciclo infinito}, 
ogni istante si ripeterà infinite volte sempre uguale. Ci sono state varie intrepretazioni. Se la 
vita è un circolo e non una linea, non ha un fine, non ha alcuno scopo. Ci sono quindi due possibili 
reazioni
\begin{enumerate}
  \item Maledire l'eterno ritorno (disperarsi) che è ciò che fa l'uomo della tradizione
  \item Vederlo come liberazione in quanto il Superuomo può vivere ogni attimo intensamente e creare
    i propri valori senza paura di un fine
\end{enumerate}
L'eterno ritorno quindi è \textbf{capace di selezionare uomini della tradizione e Superuomini}.
Probabilmente Nietsche era veramente convinto dell'eterno ritorno essendo per lui la spiegazione
più scientifica.

\subsection{``Genealogia della morale'' e ``Al di là del bene e del male''}
Da dove nascono i valori morali? Non da Dio e non dalla ragione. \textbf{I valori morali sono
  maschere} create dall'uomo. Non c'è un fondamento assoluto della morale. \textbf{Vengono determinati
nella vita sociale e variano nel tempo}. Sono una forma di controllo dell'individuo, vengon insegnati
ai bambini. Nel corso della vita si ritrovano i principi morali sottoforma di \textbf{voce della
coscienza} che in realtà è la voce del gregge, è la società con le sue idee ed insegnamenti.\\
Nietshce divide la morale in due tipologie
\begin{description}
  \item[Dei signori] È la morale degli aristocratici orgogliosi del potere. Apprezza la forza, il
    potere, ciò che arricchisce. Vede sè stesso come vicino a Dio
  \item[Degli schiavi] Nasce dal ``risentimento'' degli schiavi, dall'invidia verso i signori. 
    Apprezza i valori che il signore disprezza (umiltà, altruismo, \ldots). Si presenta come una
    morale disinteressata ma in realtà nasconde l'egoismo in modo perverso: lo schiavo pensa a sè
    come buono e voluto da Dio, al contrario del signore. Per la sua bontà verrà ricompensato da Dio,
    si sentirà superiore e proverà ``orgoglio''. Questa morale dà un senso alla vità e permette di
    superarla
\end{description}
Le due morali hanno in comune la \textbf{volontà di potenza}. Essa è
\begin{description}
  \item[Vita] Non solo come volontà di vivere ma anche come arricchimento e sviluppo
  \item[Scienza] Domina la natura, non è disinteressata, è un mezzo per controllare la natura
  \item[Morale] Apprezzare sè stessi
  \item[Arte] È creazione e nella creazione la vita si arricchisce
  \item[Superuomo] Crea nuovi valori
\end{description}
\textbf{La vita}, in quanto volontà di potenza, \textbf{è oppressione, appropriazione, imporre la 
propria volontà agli altri}. Quindi formare la vera ugugaglianza non è possibile in quanto la
democrazia è fondata sulla rinuncia alla volontà che è anche rinuncia alla vita.

\subsection{Nichilismo}
Il nichilismo nietschano è la \textbf{svalorizzazione dei valori} causata dalla morte di Dio. È la
condizione dell'uomo contemporaneo. Ce ne sono di due tipi
\begin{description}
  \item[Incompleto] I valori non sono più sostenuti, si cerca quindi di trovarne dei sostituti 
    (Positivismo) che vogliono essere assoluti
  \item[Completo] Rendersi conto della morte di Dio, non trovare dei nuovi valori. Ci sono due 
    atteggiamenti
    \begin{description}
      \item[Passivo] è quello dell'Ultimo uomo, negazione della vita e pessimismo
      \item[Attivo] è quello del Superuomo che crea dei nuovi valori
    \end{description}
\end{description}

%!TEX ROOT=filosofia.tex

\section{Kierkegaard}
Soren Aabye Kierkegaard � un filosofo \textbf{critico di Hegel}. Le sue opere principali sono
\textit{Aut-aut} e \textit{Timore e tremore}. Scriveva per difendere il cristianesimo dagli attacchi,
era critico dei luterani danesi.

\subsection{La categoria del singolo}
In Kierkegaard � fondamentale la categoria del singolo. \textbf{Quello che conta ed � reale � il 
singolo individuo, il popolo, la nazione sono tutte astrazioni.} Il valore della vita dipende 
dall'originalit� del singolo individuo. Rifiuta perci� l'idealismo e il sistemismo: racchiudere in u
unico sistema tutta la realt� � impossibile e insensato.

\subsection{La possibilit�}
Centrale in Kierkegaard � il tema della scelta. La scelta � un \textbf{salto nel vuoto}, la scelta
ci mette di fronte al nulla. Le possibilit� non scelte resteranno nel nulla. Ci sono 3 possibilit�
di fondo, o stadi dell'esistenza
\begin{description}
  \item[Esistenza estetica] Don Giovanni � preso a riferimento. La vita � dedicata al piacere e al
    godimento. Si vive nell'attimo, si vuole evitare la ripetizione. Il godimento � fisico (sessuale)
    e psicologico (della conquista del potere). � destinata alla disperazione in quanto non ha una
    continuit� e un'identit�.
  \item[Esistenza Etica] Giudice Guglielmo � il personaggio. � una vita guidata da valori morali ed
    etici. � marito (continuit�), padre, ha un lavoro onesto. Ha una storia e una personalit�. 
    Giunger� alla tristezza in quando adeguandosi ai valori morali, si uniformer� alla comunit�,
    rifiutando la singolarit�. Si pentir� dei suoi errori.
  \item[Esistenza Religiosa] Abramo � il riferimento. Deve scegliere se sacrificare Isacco, l'ordine
    di Dio � contro la morale, � una scelta irrazionale. La fede quindi � abbandonarsi a Dio senza
    sicurezze e garanzie. � una scelta individuale. Agamennone deve sacrificare Ifigenia. La 
    situazione � diversa perch� ne parla con altri e la scelta � comprensibile (sacrificare la figlia
    per un bene maggiore).
\end{description}
Questi tre stadi non sono compatibili fra di loro. Sono mutualmente esclusivi.

\subsection{L'angoscia}
L'angoscai � la percezione del nulla prima di una scelta. Non � paura. Quando scegliamo siamo di 
fronte al nulla e non ci sono garanzie che la scelta sia giusta. Questa libert� pu� portare al 
peccato.

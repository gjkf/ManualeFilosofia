%!TEX ROOT=filosofia.tex

\section{Kierkegaard}
Soren Aabye Kierkegaard è un filosofo \textbf{critico di Hegel}. Le sue opere principali sono
\textit{Aut-aut} e \textit{Timore e tremore}. Scriveva per difendere il cristianesimo dagli attacchi,
era critico dei luterani danesi.

\subsection{La categoria del singolo}
In Kierkegaard è fondamentale la categoria del singolo. \textbf{Quello che conta ed è reale è il 
singolo individuo, il popolo, la nazione sono tutte astrazioni.} Il valore della vita dipende 
dall'originalità del singolo individuo. Rifiuta perciò l'idealismo e il sistemismo: racchiudere in u
unico sistema tutta la realtà è impossibile e insensato.

\subsection{La possibilità}
Centrale in Kierkegaard è il tema della scelta. La scelta è un \textbf{salto nel vuoto}, la scelta
ci mette di fronte al nulla. Le possibilità non scelte resteranno nel nulla. Ci sono 3 possibilità
di fondo, o stadi dell'esistenza
\begin{description}
  \item[Esistenza estetica] Don Giovanni è preso a riferimento. La vita è dedicata al piacere e al
    godimento. Si vive nell'attimo, si vuole evitare la ripetizione. Il godimento è fisico (sessuale)
    e psicologico (della conquista del potere). È destinata alla disperazione in quanto non ha una
    continuità e un'identità.
  \item[Esistenza Etica] Giudice Guglielmo è il personaggio. È una vita guidata da valori morali ed
    etici. È marito (continuità), padre, ha un lavoro onesto. Ha una storia e una personalità. 
    Giungerà alla tristezza in quando adeguandosi ai valori morali, si uniformerà alla comunità,
    rifiutando la singolarità. Si pentirà dei suoi errori.
  \item[Esistenza Religiosa] Abramo è il riferimento. Deve scegliere se sacrificare Isacco, l'ordine
    di Dio è contro la morale, è una scelta irrazionale. La fede quindi è abbandonarsi a Dio senza
    sicurezze e garanzie. È una scelta individuale. Agamennone deve sacrificare Ifigenia. La 
    situazione è diversa perché ne parla con altri e la scelta è comprensibile (sacrificare la figlia
    per un bene maggiore).
\end{description}
Questi tre stadi non sono compatibili fra di loro. Sono mutualmente esclusivi.

\subsection{L'angoscia}
L'angoscai è la percezione del nulla prima di una scelta. Non è paura. Quando scegliamo siamo di 
fronte al nulla e non ci sono garanzie che la scelta sia giusta. Questa libertà può portare al 
peccato.
